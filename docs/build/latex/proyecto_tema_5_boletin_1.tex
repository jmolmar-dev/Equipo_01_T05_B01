%% Generated by Sphinx.
\def\sphinxdocclass{report}
\documentclass[a4paper,10pt,spanish]{sphinxmanual}
\ifdefined\pdfpxdimen
   \let\sphinxpxdimen\pdfpxdimen\else\newdimen\sphinxpxdimen
\fi \sphinxpxdimen=.75bp\relax
\ifdefined\pdfimageresolution
    \pdfimageresolution= \numexpr \dimexpr1in\relax/\sphinxpxdimen\relax
\fi
%% let collapsible pdf bookmarks panel have high depth per default
\PassOptionsToPackage{bookmarksdepth=5}{hyperref}

\PassOptionsToPackage{booktabs}{sphinx}
\PassOptionsToPackage{colorrows}{sphinx}

\PassOptionsToPackage{warn}{textcomp}
\usepackage[utf8]{inputenc}
\ifdefined\DeclareUnicodeCharacter
% support both utf8 and utf8x syntaxes
  \ifdefined\DeclareUnicodeCharacterAsOptional
    \def\sphinxDUC#1{\DeclareUnicodeCharacter{"#1}}
  \else
    \let\sphinxDUC\DeclareUnicodeCharacter
  \fi
  \sphinxDUC{00A0}{\nobreakspace}
  \sphinxDUC{2500}{\sphinxunichar{2500}}
  \sphinxDUC{2502}{\sphinxunichar{2502}}
  \sphinxDUC{2514}{\sphinxunichar{2514}}
  \sphinxDUC{251C}{\sphinxunichar{251C}}
  \sphinxDUC{2572}{\textbackslash}
\fi
\usepackage{cmap}
\usepackage[T1]{fontenc}
\usepackage{amsmath,amssymb,amstext}
\usepackage{babel}


\usepackage{times}


\usepackage[Sonny]{fncychap}
\ChNameVar{\Large\normalfont\sffamily}
\ChTitleVar{\Large\normalfont\sffamily}
\usepackage{sphinx}

\fvset{fontsize=auto}
\usepackage{geometry}


% Include hyperref last.
\usepackage{hyperref}
% Fix anchor placement for figures with captions.
\usepackage{hypcap}% it must be loaded after hyperref.
% Set up styles of URL: it should be placed after hyperref.
\urlstyle{same}

\addto\captionsspanish{\renewcommand{\contentsname}{Índice del Proyecto:}}

\usepackage{sphinxmessages}
\setcounter{tocdepth}{5}
\setcounter{secnumdepth}{5}

        \setcounter{secnumdepth}{0}  % Esto elimina la numeración de las secciones
    

\title{Proyecto Tema 5 Boletín 1}
\date{12 de enero de 2025}
\release{1}
\author{Juan Molero Marín, Julio Rodriguez Lopez, Julian Martinez Lorda}
\newcommand{\sphinxlogo}{\vbox{}}
\renewcommand{\releasename}{Versión}
\makeindex
\begin{document}

\ifdefined\shorthandoff
  \ifnum\catcode`\=\string=\active\shorthandoff{=}\fi
  \ifnum\catcode`\"=\active\shorthandoff{"}\fi
\fi

\pagestyle{empty}
\maketitle\newpage
\pagestyle{plain}

\pagestyle{normal}
\phantomsection\label{\detokenize{index::doc}}


\sphinxAtStartPar
Bienvenido a la documentación de nuestro proyecto realizado en el tema 5 para la asignatura de Desarrollo de Interfaces.

\sphinxAtStartPar
En esta guía se detallará toda la información necesaria para poder utilizar el framework Sphinx y poder documentar todo tu proyecto de una forma clara y concisa. Además, también generaremos documentación de código con el formato docstring de Google:

\sphinxstepscope


\chapter{1. Configuración Inicial del Proyecto}
\label{\detokenize{1_configuracion_inicial/index:configuracion-inicial-del-proyecto}}\label{\detokenize{1_configuracion_inicial/index::doc}}
\sphinxstepscope


\section{1.1 Creación del Entorno Virtual:}
\label{\detokenize{1_configuracion_inicial/entorno:creacion-del-entorno-virtual}}\label{\detokenize{1_configuracion_inicial/entorno::doc}}
\sphinxAtStartPar
Nuestro primer paso dentro de este proyecto será la creación de un Entorno Virtual en Python. Para comenzar, ejecutaremos el siguiente comando:

\begin{sphinxVerbatim}[commandchars=\\\{\}]
python\PYG{+w}{ }\PYGZhy{}m\PYG{+w}{ }venv\PYG{+w}{ }.env
\end{sphinxVerbatim}

\sphinxAtStartPar
Mediante \sphinxstylestrong{python \sphinxhyphen{}m venv}, crearemos un \sphinxstylestrong{\sphinxstyleemphasis{entorno virtual}} en el directorio donde estemos situados, en este caso, suele ser la raíz del proyecto, y posteriormente le indicamos el nombre que deseamos que tenga el mismo, en nuestro caso concreto, le hemos asignado \sphinxstylestrong{.env}.

\sphinxAtStartPar
A continuación, nuestro siguiente paso será el de instalar las \sphinxstylestrong{Librerias} que vamos a necesitar para poder realizar en el proyecto. Para ello, vamos a dirigirnos al siguiente apartado.

\sphinxstepscope


\section{1.2 Instalación y Funcionamiento de las Librerías necesarias:}
\label{\detokenize{1_configuracion_inicial/libreria:instalacion-y-funcionamiento-de-las-librerias-necesarias}}\label{\detokenize{1_configuracion_inicial/libreria::doc}}
\sphinxAtStartPar
Para la realización de este proyecto de documentación, hemos necesitado las siguientes librerías:


\begin{savenotes}\sphinxattablestart
\sphinxthistablewithglobalstyle
\centering
\begin{tabulary}{\linewidth}[t]{TT}
\sphinxtoprule
\sphinxstyletheadfamily 
\sphinxAtStartPar
\sphinxstylestrong{Librería}
&\sphinxstyletheadfamily 
\sphinxAtStartPar
\sphinxstylestrong{Descripción}
\\
\sphinxmidrule
\sphinxtableatstartofbodyhook
\sphinxAtStartPar
\sphinxstylestrong{Sphinx}
&
\sphinxAtStartPar
Herramienta para generar documentación en formatos como HTML o PDF. Soporta archivos en reStructuredText o Markdown, y es ideal para la documentación de proyectos de software.
\\
\sphinxhline
\sphinxAtStartPar
\sphinxstylestrong{Sphinx\sphinxhyphen{}rtd\sphinxhyphen{}theme}
&
\sphinxAtStartPar
Permite aplicar el tema Read the Docs en Sphinx, proporcionando un diseño limpio y profesional. Es utilizado como estándar en la documentación.
\\
\sphinxhline
\sphinxAtStartPar
\sphinxstylestrong{MyST\sphinxhyphen{}Parser}
&
\sphinxAtStartPar
Extiende a Sphinx para procesar archivos Markdown, además de reStructuredText, ofreciendo mayor flexibilidad a los usuarios.
\\
\sphinxhline
\sphinxAtStartPar
\sphinxstylestrong{Sphinxcontrib\sphinxhyphen{}napoleon}
&
\sphinxAtStartPar
Extensión que facilita la integración de docstrings escritos en los estilos de Google o Numpy. Convierte estos formatos automáticamente al adecuado para el framework.
\\
\sphinxbottomrule
\end{tabulary}
\sphinxtableafterendhook\par
\sphinxattableend\end{savenotes}

\sphinxAtStartPar
Para la instalación, se ha efectuado el siguiente comando:

\begin{sphinxVerbatim}[commandchars=\\\{\}]
pip\PYG{+w}{ }install\PYG{+w}{ }sphinx\PYG{+w}{ }sphinx\PYGZhy{}rtd\PYGZhy{}theme\PYG{+w}{ }myst\PYGZhy{}parser\PYG{+w}{ }sphinxcontrib\PYGZhy{}napoleon\PYG{+w}{ }
\end{sphinxVerbatim}

\sphinxAtStartPar
Cabe destacar que debemos de tener el \sphinxstylestrong{Entorno Virtual} activado en el momento de instalar dichas Librerías. Una vez se hayan instalado, el siguiente paso será plasmarlo dentro de un archivo de requisitos, \sphinxstyleemphasis{\sphinxstylestrong{requirements.txt}}, con el objetivo de que cualquiera pueda emplearlo en el futuro para poder realizar el mismo proyecto. Para eso se crea el archivo de texto plano en la raíz del proyecto, y posteriormente ejecutamos:

\begin{sphinxVerbatim}[commandchars=\\\{\}]
pip\PYG{+w}{ }freeze\PYG{+w}{ }\PYGZgt{}\PYG{+w}{ }requirements.txt
\end{sphinxVerbatim}

\sphinxstepscope


\section{1.3. Creación inicial del proyecto con Sphinx:}
\label{\detokenize{1_configuracion_inicial/proyecto_inicial_sphynx:creacion-inicial-del-proyecto-con-sphinx}}\label{\detokenize{1_configuracion_inicial/proyecto_inicial_sphynx::doc}}
\sphinxAtStartPar
Para inicializar un proyecto con Sphinx, una vez tengamos nuestro entorno virtual creado y las librerias necesarias instaladas, es momento de efectuar el siguiente comando por la terminal:

\begin{sphinxVerbatim}[commandchars=\\\{\}]
sphinx\PYGZhy{}quickstart\PYG{+w}{ }docs\PYG{+w}{    }
\end{sphinxVerbatim}

\sphinxAtStartPar
El parámetro \sphinxstylestrong{docs} hace referencia al directorio donde se quiere ejecutar, en nuestro caso hemos creado y escogido este debido a que es una forma de separar la \sphinxstylestrong{documentación} que vamos a generar del directorio donde posteriormente se desarrollará código python, \sphinxstylestrong{src}. Tras ejecutarlo, comenzará un proceso donde se tendrá que responder a las siguientes cuestiones:

\sphinxAtStartPar
\sphinxstylestrong{1. Lo primero será elegir entre separar o no los directorios entre el código fuente (source) y los archivos que se van a generar en base al mismo (build). En nuestro caso concreto, hemos seleccionado la opción \sphinxstyleemphasis{“yes”}}


\bigskip\hrule\bigskip


\sphinxAtStartPar
\sphinxstylestrong{2. Nombre de nuestro proyecto}


\bigskip\hrule\bigskip


\sphinxAtStartPar
\sphinxstylestrong{3. Autor/es del proyecto}


\bigskip\hrule\bigskip


\sphinxAtStartPar
\sphinxstylestrong{4. Versión del proyecto}


\bigskip\hrule\bigskip


\sphinxAtStartPar
\sphinxstylestrong{5. Idioma del proyecto}


\bigskip\hrule\bigskip


\sphinxAtStartPar
Al finalizar con las opciones escogidas, se nos formará la \sphinxstylestrong{estructura inicial del proyecto}, que explicaremos en el siguiente apartado.

\sphinxstepscope


\section{1.4 Explicación de la Estructura inicial del proyecto:}
\label{\detokenize{1_configuracion_inicial/estructura_inicial:explicacion-de-la-estructura-inicial-del-proyecto}}\label{\detokenize{1_configuracion_inicial/estructura_inicial::doc}}
\sphinxAtStartPar
Al ejecutar el comando explicado dentro del apartado anterior, y seguir el proceso de construcción en función de las especificaciones explicadas, se nos formará la siguiente estructura de proyecto \sphinxstylestrong{(en nuestro caso justo en el directorio docs al ejecutar el comando del apartado anterior desde allí)}:

\sphinxAtStartPar
\sphinxstylestrong{1. /build:} Directorio donde se irán almacenando los archivos que genere posteriormente \sphinxstylestrong{Sphinx}, que se distinguirán en función del formato de salida dentro de distintos subdirectorios: \sphinxstylestrong{HMTL,PDF, …}


\bigskip\hrule\bigskip


\sphinxAtStartPar
\sphinxstylestrong{2. /source:} Directorio principal donde se almacenarán los archivos fuente de la documentación, en este caso, serán en formato \sphinxstylestrong{Markdown}, así como el archivo de configuración principal del proyecto, en \sphinxstylestrong{Código Python}, respectivamente.


\bigskip\hrule\bigskip


\sphinxAtStartPar
\sphinxstylestrong{3. /\_static:} Subcarpeta dentro de \sphinxstylestrong{source} donde se incluirán recursos estáticos, tales como imágenes o archivos de CSS, entre otros.


\bigskip\hrule\bigskip


\sphinxAtStartPar
\sphinxstylestrong{4. /\_templates:} Carpeta destinada a archivos que personalicen nuestras plantillas HTML para generar la documentación.


\bigskip\hrule\bigskip


\sphinxAtStartPar
\sphinxstylestrong{5. conf.py:} Es el archivo de configuración principal de nuestro proyecto: recoge parámetros como el nombre del mismo o las extensiones que tiene habilitada, así como el tema que se va a emplear, entre otros \sphinxstylestrong{(Será explicado más detalladamente en el siguiente apartado)}.


\bigskip\hrule\bigskip


\sphinxAtStartPar
\sphinxstylestrong{6.index.rst:} Archivo donde se inicializará la documentación. Será el punto de partida para generar un índice y reconducir al usuario a los distintos apartados que desee consultar, los cuales serán desarrollados en otros archivos \sphinxstylestrong{Markdown} aparte. Cabe destacar que debe se ser renombrado de \sphinxstylestrong{.rst} a \sphinxstylestrong{.md}.


\bigskip\hrule\bigskip


\sphinxAtStartPar
\sphinxstylestrong{7. make.bat:} Script de \sphinxstylestrong{Windows} que nos permitirá ejecutar ciertos comandos de construcción de forma posterior, con los que generaremos la documentación en los distintos formatos.


\bigskip\hrule\bigskip


\sphinxAtStartPar
\sphinxstylestrong{8. Makefile:} Archivo empleado dentro de los \sphinxstylestrong{sistemas Unix (Linux/Mac)} para construir la documentación. Cumple la misma función que el \sphinxstylestrong{make.bat} en otros sistemas operativos

\sphinxstepscope


\section{1.5. Explicación de conf.py}
\label{\detokenize{1_configuracion_inicial/explicacion_confpy:explicacion-de-conf-py}}\label{\detokenize{1_configuracion_inicial/explicacion_confpy::doc}}
\sphinxAtStartPar
Nuestro archivo de configuración principal sería el siguiente:

\sphinxAtStartPar
\sphinxincludegraphics{{archivo_configuracion}.png}


\bigskip\hrule\bigskip


\sphinxAtStartPar
\sphinxstylestrong{Parámetros del archivo de configuración:}


\begin{savenotes}\sphinxattablestart
\sphinxthistablewithglobalstyle
\centering
\begin{tabulary}{\linewidth}[t]{TT}
\sphinxtoprule
\sphinxstyletheadfamily 
\sphinxAtStartPar
\sphinxstylestrong{Parámetro}
&\sphinxstyletheadfamily 
\sphinxAtStartPar
\sphinxstylestrong{Descripción}
\\
\sphinxmidrule
\sphinxtableatstartofbodyhook
\sphinxAtStartPar
\sphinxstylestrong{1. project}
&
\sphinxAtStartPar
Indica el nombre del proyecto.
\\
\sphinxhline
\sphinxAtStartPar
\sphinxstylestrong{2. copyright}
&
\sphinxAtStartPar
Se indicará una licencia en caso de querer establecerla.
\\
\sphinxhline
\sphinxAtStartPar
\sphinxstylestrong{3. author}
&
\sphinxAtStartPar
Autor(es) que han realizado el proyecto.
\\
\sphinxhline
\sphinxAtStartPar
\sphinxstylestrong{4. release}
&
\sphinxAtStartPar
Versión del proyecto. En este caso, es la \sphinxstylestrong{1} porque se estableció anteriormente.
\\
\sphinxhline
\sphinxAtStartPar
\sphinxstylestrong{5. Importaciones necesarias}
&
\sphinxAtStartPar
Módulos \sphinxstylestrong{os} y \sphinxstylestrong{sys}, necesarios para el manejo de rutas, y el tema de Sphinx que se va a emplear.
\\
\sphinxhline
\sphinxAtStartPar
\sphinxstylestrong{6. Extensiones}
&
\sphinxAtStartPar
Se utilizarán: \sphinxstylestrong{autodoc}, \sphinxstylestrong{napoleon} para docstring, \sphinxstylestrong{viewcode} para enlaces al código fuente, \sphinxstylestrong{sphinx\_rtd\_theme} y \sphinxstylestrong{myst\_parser} para soportar Markdown.
\\
\sphinxhline
\sphinxAtStartPar
\sphinxstylestrong{7. Otras configuraciones importantes}
&
\sphinxAtStartPar
Uso de \sphinxstylestrong{source\_suffix} para soportar archivos fuente (\sphinxstylestrong{rst} y \sphinxstylestrong{md}). Configuración del idioma y las rutas hacia los directorios \sphinxstylestrong{\_static} y \sphinxstylestrong{\_templates}.
\\
\sphinxbottomrule
\end{tabulary}
\sphinxtableafterendhook\par
\sphinxattableend\end{savenotes}

\sphinxstepscope


\section{1.6. Explicacion del archivo index.md}
\label{\detokenize{1_configuracion_inicial/explicacion_indexmd:explicacion-del-archivo-index-md}}\label{\detokenize{1_configuracion_inicial/explicacion_indexmd::doc}}
\sphinxAtStartPar
Nuestro archivo index.md va a representar el índice principal dentro de la Aplicación. En él, hemos incluido una breve descripción del proyecto también. Lo podemos observar a continuación:

\sphinxAtStartPar
\sphinxincludegraphics{{archivo_md}.png}

\sphinxAtStartPar
En este archivo podemos observar los siguientes elementos:

\sphinxAtStartPar
\sphinxstylestrong{1. Titulo Principal del Proyecto}


\bigskip\hrule\bigskip


\sphinxAtStartPar
\sphinxstylestrong{2. Indice que nos irá redirigiendo a cada uno de los apartados que se deben de explicar:} cabe destacar que dichos apartados/directorios poseen sus respectivos indices, que nos volverán a redirigir a nuevos archivos \sphinxstylestrong{Markdown} donde ya se explicará el contenido.


\bigskip\hrule\bigskip


\sphinxAtStartPar
\sphinxstylestrong{3.Elemento Caption:} nos ofrece una cabecera descriptiva para el proyecto. Se ha empleado justo antes de la generación del índice.


\bigskip\hrule\bigskip


\sphinxAtStartPar
\sphinxstylestrong{4.Elemento Maxdepth:} elemento que nos define el nivel de profundidad que podrá adquirir nuestro árbol en cuanto a directorios y subdirectorios. En nuestro caso, hemos definido el nivel 6 o nivel 2 en otros índices posteriores.


\bigskip\hrule\bigskip


\sphinxAtStartPar
El resto de elementos que podemos observar dentro de la página es \sphinxstylestrong{texto plano}, en el cual podemos plasmar la información deseada.

\sphinxstepscope


\section{1.7. Creación de ficheros de código y generación automática de documentación}
\label{\detokenize{1_configuracion_inicial/ficheros_documentacion:creacion-de-ficheros-de-codigo-y-generacion-automatica-de-documentacion}}\label{\detokenize{1_configuracion_inicial/ficheros_documentacion::doc}}
\sphinxAtStartPar
Para generar documentación en base a un archivo de código Python, tendremos que realizar los siguientes pasos:


\bigskip\hrule\bigskip


\sphinxAtStartPar
\sphinxstylestrong{1. Lo primero es la creación de un archivo de código que deseemos tranformar posteriormente en documentación mediante Sphinx. Para ello, nosotros en nuestro caso ubicamos un directorio /src a la altura de la raíz, donde nos encontramos con el archivo operaciones.py, respectivamente, que es donde se almacena nuestro código. Cabe destacar que posee el formato docstring con estilo Google:}


\bigskip\hrule\bigskip


\sphinxAtStartPar
\sphinxincludegraphics{{archivo_python}.png}


\bigskip\hrule\bigskip


\sphinxAtStartPar
\sphinxstylestrong{2.Una vez tengamos nuestro archivo listo, debemos de ejecutar el siguiente comando por la terminal. Cabe destacar que en nuestro caso concreto fue desde el directorio raíz:}

\begin{sphinxVerbatim}[commandchars=\\\{\}]
\PYG{+w}{    }sphinx\PYGZhy{}apidoc\PYG{+w}{ }\PYGZhy{}o\PYG{+w}{ }./docs\PYG{+w}{ }src\PYG{+w}{      }
\end{sphinxVerbatim}

\sphinxAtStartPar
En el cual:


\begin{savenotes}\sphinxattablestart
\sphinxthistablewithglobalstyle
\centering
\begin{tabulary}{\linewidth}[t]{TT}
\sphinxtoprule
\sphinxstyletheadfamily 
\sphinxAtStartPar
Componente
&\sphinxstyletheadfamily 
\sphinxAtStartPar
Descripción
\\
\sphinxmidrule
\sphinxtableatstartofbodyhook
\sphinxAtStartPar
\sphinxstylestrong{sphinx\sphinxhyphen{}apidoc}
&
\sphinxAtStartPar
Genera documentación automáticamente a partir de módulos de Python.
\\
\sphinxhline
\sphinxAtStartPar
\sphinxstylestrong{o ./docs}
&
\sphinxAtStartPar
Especifica el directorio donde queremos almacenar la información generada.
\\
\sphinxhline
\sphinxAtStartPar
\sphinxstylestrong{src}
&
\sphinxAtStartPar
Indica el directorio de donde se debe obtener el código y los módulos que queremos documentar.
\\
\sphinxbottomrule
\end{tabulary}
\sphinxtableafterendhook\par
\sphinxattableend\end{savenotes}


\bigskip\hrule\bigskip


\sphinxAtStartPar
\sphinxstylestrong{3.Tras la ejecución de dicho comando, se nos van a generar dos archivos rst, que hemos pasado a Markdown (operaciones y src). En nuestro caso concreto hemos eliminado el de operaciones y hemos dejado solo el de src, ya que simplemente con el podemos redigir el flujo desde el índice a la nueva documentación genera. Para ello en este mismo archivo Markdown hacemos referencia a src.md, y dentro del mismo, ya enlazamos con la información que nos encontramos dentro de operaciones:}

\sphinxAtStartPar
\sphinxincludegraphics{{flujo1}.png}


\bigskip\hrule\bigskip


\sphinxAtStartPar
\sphinxincludegraphics{{flujo2}.png}

\sphinxAtStartPar
De esta forma, desde este apartado podremos acceder a la documentación que acabamos de generar directamente, la cual vamos a mostrar a continuación


\bigskip\hrule\bigskip


\sphinxAtStartPar
\sphinxstylestrong{Documentación Generada:}

\sphinxAtStartPar
\sphinxincludegraphics{{documentacion}.png}

\sphinxstepscope
\index{module@\spxentry{module}!src.operaciones@\spxentry{src.operaciones}}\index{src.operaciones@\spxentry{src.operaciones}!module@\spxentry{module}}\index{division() (en el módulo src.operaciones)@\spxentry{division()}\spxextra{en el módulo src.operaciones}}\phantomsection\label{\detokenize{1_configuracion_inicial/src:module-src.operaciones}}

\begin{fulllineitems}
\phantomsection\label{\detokenize{1_configuracion_inicial/src:src.operaciones.division}}
\pysigstartsignatures
\pysiglinewithargsret
{\sphinxcode{\sphinxupquote{src.operaciones.}}\sphinxbfcode{\sphinxupquote{division}}}
{\sphinxparam{\DUrole{n}{a}}\sphinxparamcomma \sphinxparam{\DUrole{n}{b}}}
{}
\pysigstopsignatures
\sphinxAtStartPar
Divide dos números.
\begin{quote}\begin{description}
\sphinxlineitem{Parámetros}\begin{itemize}
\item {} 
\sphinxAtStartPar
\sphinxstyleliteralstrong{\sphinxupquote{a}} (\sphinxstyleliteralemphasis{\sphinxupquote{float}}) \textendash{} Dividendo.

\item {} 
\sphinxAtStartPar
\sphinxstyleliteralstrong{\sphinxupquote{b}} (\sphinxstyleliteralemphasis{\sphinxupquote{float}}) \textendash{} Divisor. No puede ser cero.

\end{itemize}

\sphinxlineitem{Devuelve}
\sphinxAtStartPar
Resultado de la división.

\sphinxlineitem{Tipo del valor devuelto}
\sphinxAtStartPar
float

\sphinxlineitem{Muestra}
\sphinxAtStartPar
\sphinxstyleliteralstrong{\sphinxupquote{ValueError}} \textendash{} Si el divisor es cero.

\end{description}\end{quote}

\end{fulllineitems}

\index{multiplicacion() (en el módulo src.operaciones)@\spxentry{multiplicacion()}\spxextra{en el módulo src.operaciones}}

\begin{fulllineitems}
\phantomsection\label{\detokenize{1_configuracion_inicial/src:src.operaciones.multiplicacion}}
\pysigstartsignatures
\pysiglinewithargsret
{\sphinxcode{\sphinxupquote{src.operaciones.}}\sphinxbfcode{\sphinxupquote{multiplicacion}}}
{\sphinxparam{\DUrole{n}{a}}\sphinxparamcomma \sphinxparam{\DUrole{n}{b}}}
{}
\pysigstopsignatures
\sphinxAtStartPar
Multiplica dos números.
\begin{quote}\begin{description}
\sphinxlineitem{Parámetros}\begin{itemize}
\item {} 
\sphinxAtStartPar
\sphinxstyleliteralstrong{\sphinxupquote{a}} (\sphinxstyleliteralemphasis{\sphinxupquote{float}}) \textendash{} Primer número.

\item {} 
\sphinxAtStartPar
\sphinxstyleliteralstrong{\sphinxupquote{b}} (\sphinxstyleliteralemphasis{\sphinxupquote{float}}) \textendash{} Segundo número.

\end{itemize}

\sphinxlineitem{Devuelve}
\sphinxAtStartPar
Resultado de la multiplicación.

\sphinxlineitem{Tipo del valor devuelto}
\sphinxAtStartPar
float

\end{description}\end{quote}

\end{fulllineitems}

\index{operaciones\_basicas() (en el módulo src.operaciones)@\spxentry{operaciones\_basicas()}\spxextra{en el módulo src.operaciones}}

\begin{fulllineitems}
\phantomsection\label{\detokenize{1_configuracion_inicial/src:src.operaciones.operaciones_basicas}}
\pysigstartsignatures
\pysiglinewithargsret
{\sphinxcode{\sphinxupquote{src.operaciones.}}\sphinxbfcode{\sphinxupquote{operaciones\_basicas}}}
{\sphinxparam{\DUrole{n}{a}}\sphinxparamcomma \sphinxparam{\DUrole{n}{b}}}
{}
\pysigstopsignatures
\sphinxAtStartPar
Realiza todas las operaciones básicas entre dos números.
\begin{quote}\begin{description}
\sphinxlineitem{Parámetros}\begin{itemize}
\item {} 
\sphinxAtStartPar
\sphinxstyleliteralstrong{\sphinxupquote{a}} (\sphinxstyleliteralemphasis{\sphinxupquote{float}}) \textendash{} Primer número.

\item {} 
\sphinxAtStartPar
\sphinxstyleliteralstrong{\sphinxupquote{b}} (\sphinxstyleliteralemphasis{\sphinxupquote{float}}) \textendash{} Segundo número.

\end{itemize}

\sphinxlineitem{Devuelve}
\sphinxAtStartPar
Diccionario con los resultados de suma, resta, multiplicación y división.

\sphinxlineitem{Tipo del valor devuelto}
\sphinxAtStartPar
dict

\sphinxlineitem{Muestra}
\sphinxAtStartPar
\sphinxstyleliteralstrong{\sphinxupquote{ValueError}} \textendash{} Si el divisor es cero al intentar dividir.

\end{description}\end{quote}

\end{fulllineitems}

\index{resta() (en el módulo src.operaciones)@\spxentry{resta()}\spxextra{en el módulo src.operaciones}}

\begin{fulllineitems}
\phantomsection\label{\detokenize{1_configuracion_inicial/src:src.operaciones.resta}}
\pysigstartsignatures
\pysiglinewithargsret
{\sphinxcode{\sphinxupquote{src.operaciones.}}\sphinxbfcode{\sphinxupquote{resta}}}
{\sphinxparam{\DUrole{n}{a}}\sphinxparamcomma \sphinxparam{\DUrole{n}{b}}}
{}
\pysigstopsignatures
\sphinxAtStartPar
Resta dos números.
\begin{quote}\begin{description}
\sphinxlineitem{Parámetros}\begin{itemize}
\item {} 
\sphinxAtStartPar
\sphinxstyleliteralstrong{\sphinxupquote{a}} (\sphinxstyleliteralemphasis{\sphinxupquote{float}}) \textendash{} Primer número.

\item {} 
\sphinxAtStartPar
\sphinxstyleliteralstrong{\sphinxupquote{b}} (\sphinxstyleliteralemphasis{\sphinxupquote{float}}) \textendash{} Segundo número.

\end{itemize}

\sphinxlineitem{Devuelve}
\sphinxAtStartPar
Resultado de la resta.

\sphinxlineitem{Tipo del valor devuelto}
\sphinxAtStartPar
float

\end{description}\end{quote}

\end{fulllineitems}

\index{suma() (en el módulo src.operaciones)@\spxentry{suma()}\spxextra{en el módulo src.operaciones}}

\begin{fulllineitems}
\phantomsection\label{\detokenize{1_configuracion_inicial/src:src.operaciones.suma}}
\pysigstartsignatures
\pysiglinewithargsret
{\sphinxcode{\sphinxupquote{src.operaciones.}}\sphinxbfcode{\sphinxupquote{suma}}}
{\sphinxparam{\DUrole{n}{a}}\sphinxparamcomma \sphinxparam{\DUrole{n}{b}}}
{}
\pysigstopsignatures
\sphinxAtStartPar
Suma dos números.
\begin{quote}\begin{description}
\sphinxlineitem{Parámetros}\begin{itemize}
\item {} 
\sphinxAtStartPar
\sphinxstyleliteralstrong{\sphinxupquote{a}} (\sphinxstyleliteralemphasis{\sphinxupquote{float}}) \textendash{} Primer número.

\item {} 
\sphinxAtStartPar
\sphinxstyleliteralstrong{\sphinxupquote{b}} (\sphinxstyleliteralemphasis{\sphinxupquote{float}}) \textendash{} Segundo número.

\end{itemize}

\sphinxlineitem{Devuelve}
\sphinxAtStartPar
Resultado de la suma.

\sphinxlineitem{Tipo del valor devuelto}
\sphinxAtStartPar
float

\end{description}\end{quote}

\end{fulllineitems}


\sphinxstepscope


\section{1.8. Generación de HTML a partir de la documentación}
\label{\detokenize{1_configuracion_inicial/generacion_html:generacion-de-html-a-partir-de-la-documentacion}}\label{\detokenize{1_configuracion_inicial/generacion_html::doc}}
\sphinxAtStartPar
Una vez tengamos todo listo para generar nuestra documentación, será momento de efectuar el siguiente comando por consola:

\begin{sphinxVerbatim}[commandchars=\\\{\}]
\PYG{+w}{    }sphinx\PYGZhy{}build\PYG{+w}{ }\PYGZhy{}b\PYG{+w}{ }html\PYG{+w}{ }.\PYG{+w}{ }../build/html\PYG{+w}{ }
\end{sphinxVerbatim}

\sphinxAtStartPar
Donde:


\begin{savenotes}\sphinxattablestart
\sphinxthistablewithglobalstyle
\centering
\begin{tabulary}{\linewidth}[t]{TT}
\sphinxtoprule
\sphinxstyletheadfamily 
\sphinxAtStartPar
\sphinxstylestrong{Parámetro/Parte}
&\sphinxstyletheadfamily 
\sphinxAtStartPar
\sphinxstylestrong{Descripción}
\\
\sphinxmidrule
\sphinxtableatstartofbodyhook
\sphinxAtStartPar
\sphinxstylestrong{sphinx\sphinxhyphen{}build}
&
\sphinxAtStartPar
Herramienta de Sphinx con la cual generaremos la documentación.
\\
\sphinxhline
\sphinxAtStartPar
\sphinxstylestrong{\sphinxhyphen{}b}
&
\sphinxAtStartPar
Indicador para especificar el formato en el cual se generará la documentación.
\\
\sphinxhline
\sphinxAtStartPar
\sphinxstylestrong{html}
&
\sphinxAtStartPar
Formato respectivo en el que vamos a generar la documentación.
\\
\sphinxhline
\sphinxAtStartPar
\sphinxstylestrong{.}
&
\sphinxAtStartPar
Directorio actual donde nos encontramos. En este caso, se ejecuta desde la carpeta source.
\\
\sphinxhline
\sphinxAtStartPar
\sphinxstylestrong{../build/html}
&
\sphinxAtStartPar
Ruta hacia el directorio donde queremos que se guarden los archivos resultantes.
\\
\sphinxbottomrule
\end{tabulary}
\sphinxtableafterendhook\par
\sphinxattableend\end{savenotes}


\bigskip\hrule\bigskip


\sphinxAtStartPar
Una vez efectuado el comando, se nos generará nuestra documentación en HTML respectivamente, dentro del subdirectorio \sphinxstylestrong{html} ubicado en \sphinxstylestrong{/build}. Aunque se nos generán varios archivos, realmente nos centraremos en el \sphinxstyleemphasis{\sphinxstylestrong{index.html}}, el cual si lo visualizamos podemos observar como ya contendría el índice principal con el cual podemos acceder a nuestra documentación:

\sphinxAtStartPar
\sphinxincludegraphics{{indice}.png}

\sphinxstepscope


\section{1.9. Generación de PDF a partir de la documentación:}
\label{\detokenize{1_configuracion_inicial/generacion_pdf:generacion-de-pdf-a-partir-de-la-documentacion}}\label{\detokenize{1_configuracion_inicial/generacion_pdf::doc}}
\sphinxAtStartPar
Para la generación de archivos PDF, en nuestro caso hemos escogido el sistema que emplea la tipografía LaTeX, mediante la cual gracias a un intérprete vamos a poder generar archivos en formato \sphinxstylestrong{.tex}, que posteriormente se compilarán en documentos \sphinxstylestrong{PDF} con una apareriencia profesional. Para ello, el primer paso será tener una distribución de TeX propiamente instalada dentro del equipo donde se quiera realizar esta tarea. Es por eso que en nuestro caso hemos decidido instalar \sphinxstylestrong{MiKTeX}. El proceso de instalación será el siguiente:

\sphinxAtStartPar
\sphinxstylestrong{1. Buscamos MiKTeX en nuestro navegador e indicamos en que parte del sistema deseamos instalarlo}


\bigskip\hrule\bigskip


\sphinxAtStartPar
\sphinxstylestrong{2. Al instalarlo, abriremos posteriormente su consola para actualizarlo y agregaremos los paquetes de latexpdf necesarios para generar nuestros documentos}


\bigskip\hrule\bigskip


\sphinxAtStartPar
\sphinxstylestrong{3. Finalmente, indicaremos al Sistema Operativo mediante la modificación de la Variable del Sistema Path la ruta de instalación de MiKTeX, con la finalidad de que se reconozca posteriormente el comando que tenemos que emplear}


\bigskip\hrule\bigskip


\sphinxAtStartPar
Tras realizar todos estos pasos, es momento de dirigirnos a nuestro proyecto y comenzar a generar la documentación. Para eso ejecutaremos el siguiente comando:

\begin{sphinxVerbatim}[commandchars=\\\{\}]
\PYG{o}{.}\PYGZbs{}\PYG{n}{make}\PYG{o}{.}\PYG{n}{bat} \PYG{n}{clean}\PYG{p}{;} \PYG{o}{.}\PYGZbs{}\PYG{n}{make}\PYG{o}{.}\PYG{n}{bat} \PYG{n}{latexpdf}
\end{sphinxVerbatim}


\begin{savenotes}\sphinxattablestart
\sphinxthistablewithglobalstyle
\centering
\begin{tabulary}{\linewidth}[t]{TT}
\sphinxtoprule
\sphinxstyletheadfamily 
\sphinxAtStartPar
\sphinxstylestrong{Comando}
&\sphinxstyletheadfamily 
\sphinxAtStartPar
\sphinxstylestrong{Explicación}
\\
\sphinxmidrule
\sphinxtableatstartofbodyhook
\sphinxAtStartPar
\sphinxstylestrong{.\textbackslash{}make.bat clean}
&
\sphinxAtStartPar
Ejecuta el script \sphinxcode{\sphinxupquote{make.bat}} con la opción \sphinxcode{\sphinxupquote{clean}}, lo que significa que limpia cualquier archivo o compilación previa en el proyecto, eliminando archivos generados anteriormente.
\\
\sphinxhline
\sphinxAtStartPar
\sphinxstylestrong{.\textbackslash{}make.bat latexpdf}
&
\sphinxAtStartPar
Ejecuta el script \sphinxcode{\sphinxupquote{make.bat}} con la opción \sphinxcode{\sphinxupquote{latexpdf}}, que compila la documentación en formato LaTeX y la convierte en un archivo PDF.
\\
\sphinxbottomrule
\end{tabulary}
\sphinxtableafterendhook\par
\sphinxattableend\end{savenotes}

\sphinxAtStartPar
Al ejecutar el comando, la carpeta build se nos reestructurará y se nos generará el archivo \sphinxstylestrong{.pdf} correspondiente dentro del subdirectorio \sphinxstylestrong{/latex}.


\bigskip\hrule\bigskip


\sphinxAtStartPar
\sphinxstylestrong{IMPORTANTE}
\begin{itemize}
\item {} 
\sphinxAtStartPar
Para que el formato \sphinxstylestrong{PDF} salga con el formato deseado y con el fin de evitar errores, como duplicación de páginas en blanco o de índices en la numeración, se ha tenido que agregar la siguiente configuración dentro del archivo \sphinxstylestrong{conf.py}:

\end{itemize}

\sphinxAtStartPar
\sphinxincludegraphics{{nueva_configuracion}.png}

\sphinxstepscope


\section{1.10 Explicación de la Estructura Final del Proyecto:}
\label{\detokenize{1_configuracion_inicial/estructura_final:explicacion-de-la-estructura-final-del-proyecto}}\label{\detokenize{1_configuracion_inicial/estructura_final::doc}}
\sphinxAtStartPar
\sphinxstylestrong{RESUMEN ESTRUCTURA INICIAL}:


\begin{savenotes}\sphinxattablestart
\sphinxthistablewithglobalstyle
\centering
\begin{tabulary}{\linewidth}[t]{TT}
\sphinxtoprule
\sphinxstyletheadfamily 
\sphinxAtStartPar
\sphinxstylestrong{Componente}
&\sphinxstyletheadfamily 
\sphinxAtStartPar
\sphinxstylestrong{Explicación}
\\
\sphinxmidrule
\sphinxtableatstartofbodyhook
\sphinxAtStartPar
\sphinxstylestrong{/build}
&
\sphinxAtStartPar
Directorio donde se irán almacenando los archivos que genere posteriormente \sphinxstylestrong{Sphinx}, que se distinguirán en función del formato de salida dentro de distintos subdirectorios: \sphinxstylestrong{HTML, PDF, …}
\\
\sphinxhline
\sphinxAtStartPar
\sphinxstylestrong{/source}
&
\sphinxAtStartPar
Directorio principal donde se almacenarán los archivos fuente de la documentación, en este caso, serán en formato \sphinxstylestrong{Markdown}, así como el archivo de configuración principal del proyecto, en \sphinxstylestrong{Código Python}, respectivamente.
\\
\sphinxhline
\sphinxAtStartPar
\sphinxstylestrong{/\_static}
&
\sphinxAtStartPar
Subcarpeta dentro de \sphinxstylestrong{source} donde se incluirán recursos estáticos, tales como imágenes o archivos de CSS, entre otros.
\\
\sphinxhline
\sphinxAtStartPar
\sphinxstylestrong{/\_templates}
&
\sphinxAtStartPar
Carpeta destinada a archivos que personalicen nuestras plantillas HTML para generar la documentación.
\\
\sphinxhline
\sphinxAtStartPar
\sphinxstylestrong{conf.py}
&
\sphinxAtStartPar
Es el archivo de configuración principal de nuestro proyecto: recoge parámetros como el nombre del mismo o las extensiones que tiene habilitada, así como el tema que se va a emplear, entre otros. \sphinxstylestrong{(Será explicado más detalladamente en el siguiente apartado)}.
\\
\sphinxhline
\sphinxAtStartPar
\sphinxstylestrong{index.rst}
&
\sphinxAtStartPar
Archivo donde se inicializará la documentación. Será el punto de partida para generar un índice y reconducir al usuario a los distintos apartados que desee consultar, los cuales serán desarrollados en otros archivos \sphinxstylestrong{Markdown} aparte. Cabe destacar que debe ser renombrado de \sphinxstylestrong{.rst} a \sphinxstylestrong{.md}.
\\
\sphinxhline
\sphinxAtStartPar
\sphinxstylestrong{make.bat}
&
\sphinxAtStartPar
Script de \sphinxstylestrong{Windows} que nos permitirá ejecutar ciertos comandos de construcción de forma posterior, con los que generaremos la documentación en los distintos formatos.
\\
\sphinxhline
\sphinxAtStartPar
\sphinxstylestrong{Makefile}
&
\sphinxAtStartPar
Archivo empleado dentro de los \sphinxstylestrong{sistemas Unix (Linux/Mac)} para construir la documentación. Cumple la misma función que el \sphinxstylestrong{make.bat} en otros sistemas operativos.
\\
\sphinxbottomrule
\end{tabulary}
\sphinxtableafterendhook\par
\sphinxattableend\end{savenotes}

\sphinxAtStartPar
A la estructura inicial vista en el apartado anterior, con todos los cambios que se han ido realizando punto, deberíamos incluirle los siguientes directorios/archivos:

\sphinxAtStartPar
\sphinxstylestrong{1. Subdirectorios /1\_configuracion\_inicial, 2\_comandos, 3\_guia\_myst:}

\sphinxAtStartPar
Estos subdirectorios son los que almacenarán todo el contenido en formato \sphinxstylestrong{Markdown}, con el cual hemos ido explicando y documentando todos los apartados y lo vamos a seguir haciendo hasta la finalización. Cabe destacar que cada uno de ellos posee su propio archivo \sphinxstylestrong{index.md}, con el objetivo de tener un índice y un direccionamiento más detallado dentro de la documentación.


\bigskip\hrule\bigskip


\sphinxAtStartPar
\sphinxstylestrong{2. Subdirectorios dentro del directorio /build:}

\sphinxAtStartPar
Al generar nuestra documentación, se nos generarán dos directorios dentro del mismo:
\begin{itemize}
\item {} 
\sphinxAtStartPar
En el caso de generar \sphinxstylestrong{HTML}, serán doctrees y html respectivamente.

\item {} 
\sphinxAtStartPar
Si generamos \sphinxstylestrong{PDF}, se sustituirá el de html por \sphinxstylestrong{latex}.

\end{itemize}


\bigskip\hrule\bigskip


\sphinxAtStartPar
\sphinxstylestrong{3. Directorio src:}

\sphinxAtStartPar
Directorio donde durante nuestro proyecto hemos tenido que almacenar el código \sphinxstylestrong{Python}, con el que hemos generado la documentación con el formato docstring de Google posteriormente, lo cual ya se ha explicado de forma previa.

\sphinxstepscope


\chapter{2.Comandos Utilizados de forma detallada en el proyecto}
\label{\detokenize{2_comandos/index:comandos-utilizados-de-forma-detallada-en-el-proyecto}}\label{\detokenize{2_comandos/index::doc}}
\sphinxAtStartPar
Estos serian los comandos que hemos utilizado dentro del proyecto:


\begin{savenotes}\sphinxattablestart
\sphinxthistablewithglobalstyle
\centering
\begin{tabulary}{\linewidth}[t]{TT}
\sphinxtoprule
\sphinxstyletheadfamily 
\sphinxAtStartPar
\sphinxstylestrong{Comando}
&\sphinxstyletheadfamily 
\sphinxAtStartPar
\sphinxstylestrong{Descripción}
\\
\sphinxmidrule
\sphinxtableatstartofbodyhook
\sphinxAtStartPar
\sphinxstylestrong{python \sphinxhyphen{}m venv .env}
&
\sphinxAtStartPar
Este comando se ha utilizado en el proyecto para construir el entorno virtual desde la raíz del proyecto. \sphinxcode{\sphinxupquote{.env}} indica el nombre que se asignará al entorno.
\\
\sphinxhline
\sphinxAtStartPar
\sphinxstylestrong{..env\textbackslash{}Scripts\textbackslash{}activate}
&
\sphinxAtStartPar
Este comando se ha usado para activar el entorno de trabajo.
\\
\sphinxhline
\sphinxAtStartPar
\sphinxstylestrong{pip install sphinx sphinx\sphinxhyphen{}rtd\sphinxhyphen{}theme myst\sphinxhyphen{}parser sphinxcontrib\sphinxhyphen{}napoleon}
&
\sphinxAtStartPar
Este comando instala las librerías necesarias para el proyecto: \sphinxcode{\sphinxupquote{Sphinx}}, \sphinxcode{\sphinxupquote{sphinx\sphinxhyphen{}rtd\sphinxhyphen{}theme}}, \sphinxcode{\sphinxupquote{MyST\sphinxhyphen{}Parser}} y \sphinxcode{\sphinxupquote{sphinxcontrib\sphinxhyphen{}napoleon}}.
\\
\sphinxhline
\sphinxAtStartPar
\sphinxstylestrong{pip freeze \textgreater{} requirements.txt}
&
\sphinxAtStartPar
«pip freeze» enumera todas las dependencias instaladas en el entorno virtual o global de Python. El símbolo \sphinxcode{\sphinxupquote{\textgreater{}}} redirecciona la salida al archivo \sphinxcode{\sphinxupquote{requirements.txt}}. Así, este comando guarda las dependencias del proyecto para que otros puedan instalar las mismas y ejecutar el proyecto sin problemas.
\\
\sphinxhline
\sphinxAtStartPar
\sphinxstylestrong{sphinx\sphinxhyphen{}quickstart docs}
&
\sphinxAtStartPar
Este comando inicializa el proyecto con Sphinx. El parámetro \sphinxcode{\sphinxupquote{docs}} indica el directorio donde se quiere ejecutar, en este caso, la carpeta \sphinxcode{\sphinxupquote{docs}}. Durante la ejecución, el comando solicita separar los archivos fuente (\sphinxcode{\sphinxupquote{source}}) de los generados (\sphinxcode{\sphinxupquote{build}}). Se eligió la opción \sphinxcode{\sphinxupquote{yes}}. También solicita el nombre del proyecto (indicado como «Proyecto Tema 5 Equipo 01»), los autores, la versión (1), y el idioma (español).
\\
\sphinxhline
\sphinxAtStartPar
\sphinxstylestrong{cd .\textbackslash{}docs}
&
\sphinxAtStartPar
Cambia el directorio actual a la carpeta \sphinxcode{\sphinxupquote{docs}}, donde se encuentra el archivo \sphinxcode{\sphinxupquote{make.bat}}. Este paso es necesario antes de ejecutar los siguientes comandos.
\\
\sphinxhline
\sphinxAtStartPar
\sphinxstylestrong{.\textbackslash{}make.bat clean}
&
\sphinxAtStartPar
Limpia todos los archivos temporales y resultados previos de la generación de la documentación, eliminando los contenidos de la carpeta \sphinxcode{\sphinxupquote{\_build}}.
\\
\sphinxhline
\sphinxAtStartPar
\sphinxstylestrong{.\textbackslash{}make.bat html}
&
\sphinxAtStartPar
Compila los archivos fuente de la documentación para generar la versión HTML, cuyo resultado se guarda en la carpeta \sphinxcode{\sphinxupquote{\_build/html}}.
\\
\sphinxhline
\sphinxAtStartPar
\sphinxstylestrong{sphinx\sphinxhyphen{}apidoc \sphinxhyphen{}o ./docs src}
&
\sphinxAtStartPar
Genera archivos \sphinxcode{\sphinxupquote{.rst}} en la carpeta \sphinxcode{\sphinxupquote{docs}} a partir del código fuente ubicado en la carpeta \sphinxcode{\sphinxupquote{src}}. Esto automatiza la creación de documentación para módulos y paquetes de Python.
\\
\sphinxhline
\sphinxAtStartPar
\sphinxstylestrong{sphinx\sphinxhyphen{}build \sphinxhyphen{}b html . ../build/html}
&
\sphinxAtStartPar
Genera la documentación en formato HTML desde los archivos fuente (\sphinxcode{\sphinxupquote{.}}) y guarda los resultados en el directorio \sphinxcode{\sphinxupquote{../build/html}}.
\\
\sphinxhline
\sphinxAtStartPar
\sphinxstylestrong{.\textbackslash{}make.bat latexpdf}
&
\sphinxAtStartPar
Ejecuta el script \sphinxcode{\sphinxupquote{make.bat}} con la opción \sphinxcode{\sphinxupquote{latexpdf}}, que compila la documentación en formato LaTeX y la convierte en un archivo PDF.
\\
\sphinxbottomrule
\end{tabulary}
\sphinxtableafterendhook\par
\sphinxattableend\end{savenotes}

\sphinxstepscope


\chapter{3. Guia de Uso Myst}
\label{\detokenize{3_guia_myst/index:guia-de-uso-myst}}\label{\detokenize{3_guia_myst/index::doc}}
\sphinxstepscope


\section{3.1 Tipografia MYST:}
\label{\detokenize{3_guia_myst/tipografia:tipografia-myst}}\label{\detokenize{3_guia_myst/tipografia::doc}}

\subsection{3.1.1 Encabezados}
\label{\detokenize{3_guia_myst/tipografia:encabezados}}
\sphinxAtStartPar
Los encabezados funcionan de la misma manera que en markdown, usando almohadillas consecutivas.

\begin{sphinxadmonition}{note}{Ejemplo Encabezados}

\sphinxAtStartPar
Encabezado de nivel 1

\sphinxAtStartPar
Encabezado de nivel 2

\sphinxAtStartPar
Encabezado de nivel 3

\sphinxAtStartPar
Encabezado de nivel 4
\end{sphinxadmonition}

\sphinxAtStartPar
Lo que sería el resultado de:

\begin{sphinxadmonition}{note}{Markdown}

\begin{sphinxVerbatim}[commandchars=\\\{\}]
\PYG{c+c1}{\PYGZsh{} Encabezado de nivel 1}
\PYG{c+c1}{\PYGZsh{}\PYGZsh{} Encabezado de nivel 2}
\PYG{c+c1}{\PYGZsh{}\PYGZsh{}\PYGZsh{} Encabezado de nivel 3}
\PYG{c+c1}{\PYGZsh{}\PYGZsh{}\PYGZsh{}\PYGZsh{} Encabezado de nivel 4}
\end{sphinxVerbatim}
\end{sphinxadmonition}


\bigskip\hrule\bigskip



\subsection{3.1.2 Parráfos}
\label{\detokenize{3_guia_myst/tipografia:parrafos}}
\sphinxAtStartPar
Un parrafo es una linea de texto separada por una linea en blanco.

\begin{sphinxadmonition}{note}{Ejemplo Parrafo}

\sphinxAtStartPar
Ejemplo Parrafo 1.

\sphinxAtStartPar
Ejemplo Parrafo 2.
\end{sphinxadmonition}


\bigskip\hrule\bigskip



\subsection{3.1.3 Cambios de tema}
\label{\detokenize{3_guia_myst/tipografia:cambios-de-tema}}
\sphinxAtStartPar
Un cambio de tema separa dos secciones de temas distintos.


\bigskip\hrule\bigskip


\sphinxAtStartPar
Lo que es el resultado de:

\begin{sphinxadmonition}{note}{Markdown}

\begin{sphinxVerbatim}[commandchars=\\\{\}]
\PYG{o}{\PYGZhy{}}\PYG{o}{\PYGZhy{}}\PYG{o}{\PYGZhy{}}
\end{sphinxVerbatim}
\end{sphinxadmonition}


\bigskip\hrule\bigskip



\subsection{3.1.4 Estilos de Texto}
\label{\detokenize{3_guia_myst/tipografia:estilos-de-texto}}
\begin{sphinxadmonition}{note}{Estilos de Texto}

\sphinxAtStartPar
Este es un texto \sphinxstylestrong{en negrita}.

\sphinxAtStartPar
Este es un texto \sphinxstyleemphasis{en cursiva}.

\sphinxAtStartPar
Este texto está en $_{\text{subíndice}}$.

\sphinxAtStartPar
Este texto está en $^{\text{superíndice}}$.
\end{sphinxadmonition}

\sphinxAtStartPar
Lo cual es el resultado de:

\begin{sphinxadmonition}{note}{Markdown}

\begin{sphinxVerbatim}[commandchars=\\\{\}]
Este es un texto **en negrita**.

Este es un texto *en cursiva*.

Este texto está en \PYGZob{}sub\PYGZcb{}`subíndice`.

Este texto está en \PYGZob{}sup\PYGZcb{}`superíndice`.
\end{sphinxVerbatim}
\end{sphinxadmonition}


\bigskip\hrule\bigskip



\subsection{3.1.5 Salto de linea.}
\label{\detokenize{3_guia_myst/tipografia:salto-de-linea}}
\sphinxAtStartPar
Un salto de linea separa dos lineas sin necesidad de una linea en blanco.

\begin{sphinxadmonition}{note}{Ejemplo Salto de Linea}

\sphinxAtStartPar
Pulgas \\
tenía \\
el perro.
\end{sphinxadmonition}

\sphinxAtStartPar
Lo cual es el resultado de:

\begin{sphinxadmonition}{note}{Markdown}

\begin{sphinxVerbatim}[commandchars=\\\{\}]
\PYG{n}{Pulgas} \PYGZbs{}
\PYG{n}{tenía} \PYGZbs{}
\PYG{n}{el} \PYG{n}{perro}\PYG{o}{.}
\end{sphinxVerbatim}
\end{sphinxadmonition}


\bigskip\hrule\bigskip



\subsection{3.1.6 Listas numeradas y no numeradas}
\label{\detokenize{3_guia_myst/tipografia:listas-numeradas-y-no-numeradas}}
\begin{sphinxadmonition}{note}{Ejemplo Listas Numeradas y No Numeradas}
\begin{itemize}
\item {} 
\sphinxAtStartPar
Lista no numerada empezada por \sphinxhyphen{}

\item {} 
\sphinxAtStartPar
Con dos puntos
\begin{itemize}
\item {} 
\sphinxAtStartPar
Y una lista anidada empezada por *

\end{itemize}

\end{itemize}
\begin{itemize}
\item {} 
\sphinxAtStartPar
Tambien puede hacerse a la inversa

\item {} 
\sphinxAtStartPar
Con una lista no numerada empezada por *
\begin{itemize}
\item {} 
\sphinxAtStartPar
Y listas anidadas empezadas por \sphinxhyphen{}

\end{itemize}

\item {} 
\sphinxAtStartPar
E incluso
\begin{enumerate}
\sphinxsetlistlabels{\arabic}{enumi}{enumii}{}{.}%
\item {} 
\sphinxAtStartPar
Listas numeradas

\item {} 
\sphinxAtStartPar
Anidadas

\end{enumerate}

\end{itemize}
\begin{enumerate}
\sphinxsetlistlabels{\arabic}{enumi}{enumii}{}{.}%
\item {} 
\sphinxAtStartPar
Las listas numeradas

\item {} 
\sphinxAtStartPar
Por supuesto
\begin{enumerate}
\sphinxsetlistlabels{\arabic}{enumii}{enumiii}{}{.}%
\item {} 
\sphinxAtStartPar
Pueden tener

\item {} 
\sphinxAtStartPar
Listas anidadas

\end{enumerate}

\item {} 
\sphinxAtStartPar
Y estar solas

\end{enumerate}
\end{sphinxadmonition}

\sphinxAtStartPar
Lo cual es el resultado de:

\begin{sphinxadmonition}{note}{Markdown}

\begin{sphinxVerbatim}[commandchars=\\\{\}]
\PYG{o}{\PYGZhy{}} \PYG{n}{Lista} \PYG{n}{no} \PYG{n}{numerada} \PYG{n}{empezada} \PYG{n}{por} \PYGZbs{}\PYG{o}{\PYGZhy{}} 
\PYG{o}{\PYGZhy{}} \PYG{n}{Con} \PYG{n}{dos} \PYG{n}{puntos}
    \PYG{o}{*} \PYG{n}{Y} \PYG{n}{una} \PYG{n}{lista} \PYG{n}{anidada} \PYG{n}{empezada} \PYG{n}{por} \PYGZbs{}\PYG{o}{*} 

\PYG{o}{*} \PYG{n}{Tambien} \PYG{n}{puede} \PYG{n}{hacerse} \PYG{n}{a} \PYG{n}{la} \PYG{n}{inversa}
\PYG{o}{*} \PYG{n}{Con} \PYG{n}{una} \PYG{n}{lista} \PYG{n}{no} \PYG{n}{numerada} \PYG{n}{empezada} \PYG{n}{por} \PYGZbs{}\PYG{o}{*}
    \PYG{o}{\PYGZhy{}} \PYG{n}{Y} \PYG{n}{listas} \PYG{n}{anidadas} \PYG{n}{empezadas} \PYG{n}{por} \PYGZbs{}\PYG{o}{\PYGZhy{}}
\PYG{o}{*} \PYG{n}{E} \PYG{n}{incluso}
    \PYG{l+m+mf}{1.} \PYG{n}{Listas} \PYG{n}{numeradas}
    \PYG{l+m+mf}{2.} \PYG{n}{Anidadas}

\PYG{l+m+mf}{1.} \PYG{n}{Las} \PYG{n}{listas} \PYG{n}{numeradas}
\PYG{l+m+mf}{2.} \PYG{n}{Por} \PYG{n}{supuesto}
    \PYG{l+m+mf}{1.} \PYG{n}{Pueden} \PYG{n}{tener}
    \PYG{l+m+mf}{2.} \PYG{n}{Listas} \PYG{n}{anidadas}
\PYG{l+m+mf}{3.} \PYG{n}{Y} \PYG{n}{estar} \PYG{n}{solas}
\end{sphinxVerbatim}
\end{sphinxadmonition}


\bigskip\hrule\bigskip



\subsection{3.1.7 Citas}
\label{\detokenize{3_guia_myst/tipografia:citas}}\begin{quote}

\sphinxAtStartPar
Aunque la mona se vista de seda, mona se queda.
\end{quote}

\sphinxAtStartPar
Esto es el resultado de:

\begin{sphinxadmonition}{note}{Markdown}

\begin{sphinxVerbatim}[commandchars=\\\{\}]
\PYG{o}{\PYGZgt{}} \PYG{n}{Aunque} \PYG{n}{la} \PYG{n}{mona} \PYG{n}{se} \PYG{n}{vista} \PYG{n}{de} \PYG{n}{seda}\PYG{p}{,} \PYG{n}{mona} \PYG{n}{se} \PYG{n}{queda}\PYG{o}{.}
\end{sphinxVerbatim}
\end{sphinxadmonition}


\bigskip\hrule\bigskip



\subsection{3.1.8 Notas de Pie de Página}
\label{\detokenize{3_guia_myst/tipografia:notas-de-pie-de-pagina}}
\begin{sphinxadmonition}{note}{Ejemplo Notas de Pie de Página}
\begin{itemize}
\item {} 
\sphinxAtStartPar
Esta referencia a nota de pie de página esta numerada manualmente.%
\begin{footnote}[3]\sphinxAtStartFootnote
This is a manually\sphinxhyphen{}numbered footnote definition.
%
\end{footnote}

\item {} 
\sphinxAtStartPar
Mientras que esta está numerada automaticamente.%
\begin{footnote}[1]\sphinxAtStartFootnote
This is an auto\sphinxhyphen{}numbered footnote definition.
%
\end{footnote}

\end{itemize}
\end{sphinxadmonition}

\sphinxAtStartPar
Esto es el resultado de:

\begin{sphinxadmonition}{note}{Markdown}

\begin{sphinxVerbatim}[commandchars=\\\{\}]
\PYG{o}{\PYGZhy{}} \PYG{n}{Esta} \PYG{n}{referencia} \PYG{n}{a} \PYG{n}{nota} \PYG{n}{de} \PYG{n}{pie} \PYG{n}{de} \PYG{n}{página} \PYG{n}{esta} \PYG{n}{numerada} \PYG{n}{manualmente}\PYG{o}{.}\PYG{p}{[}\PYG{o}{\PYGZca{}}\PYG{l+m+mi}{3}\PYG{p}{]}
\PYG{o}{\PYGZhy{}} \PYG{n}{Mientras} \PYG{n}{que} \PYG{n}{esta} \PYG{n}{está} \PYG{n}{numerada} \PYG{n}{automaticamente}\PYG{o}{.}\PYG{p}{[}\PYG{o}{\PYGZca{}}\PYG{n}{myref}\PYG{p}{]}
\PYG{p}{[}\PYG{o}{\PYGZca{}}\PYG{n}{myref}\PYG{p}{]}\PYG{p}{:} \PYG{n}{This} \PYG{o+ow}{is} \PYG{n}{an} \PYG{n}{auto}\PYG{o}{\PYGZhy{}}\PYG{n}{numbered} \PYG{n}{footnote} \PYG{n}{definition}\PYG{o}{.}
\PYG{p}{[}\PYG{o}{\PYGZca{}}\PYG{l+m+mi}{3}\PYG{p}{]}\PYG{p}{:} \PYG{n}{This} \PYG{o+ow}{is} \PYG{n}{a} \PYG{n}{manually}\PYG{o}{\PYGZhy{}}\PYG{n}{numbered} \PYG{n}{footnote} \PYG{n}{definition}\PYG{o}{.}
\end{sphinxVerbatim}
\end{sphinxadmonition}


\bigskip\hrule\bigskip


\sphinxstepscope


\section{3.2 Admonitions en MyST}
\label{\detokenize{3_guia_myst/advertencias:admonitions-en-myst}}\label{\detokenize{3_guia_myst/advertencias::doc}}

\subsection{3.2.1 ¿Qué son los Admonitions?}
\label{\detokenize{3_guia_myst/advertencias:que-son-los-admonitions}}
\sphinxAtStartPar
Los admonitions en MyST son bloques de contenido destacados que puedes usar para resaltar información importante, advertencias, ejemplos, o cualquier tipo de contenido que quieras destacar. MyST soporta varios tipos de admonitions predefinidos y personalizables.

\begin{sphinxadmonition}{note}{Ejemplo de Admonition Básico}

\sphinxAtStartPar
Este es un admonition básico sin tipo específico.
\end{sphinxadmonition}

\sphinxAtStartPar
Esto se logra con:

\begin{sphinxadmonition}{note}{Markdown}

\begin{sphinxVerbatim}[commandchars=\\\{\}]
```\PYGZob{}admonition\PYGZcb{} Ejemplo de Admonition Básico
Este es un admonition básico sin tipo específico.
\end{sphinxVerbatim}
\end{sphinxadmonition}


\bigskip\hrule\bigskip



\subsection{3.2.2 Admonitions Predefinidos}
\label{\detokenize{3_guia_myst/advertencias:admonitions-predefinidos}}
\sphinxAtStartPar
MyST incluye admonitions predefinidos como \sphinxcode{\sphinxupquote{note}}, \sphinxcode{\sphinxupquote{warning}}, \sphinxcode{\sphinxupquote{tip}}, y más.

\begin{sphinxadmonition}{note}{Ejemplo de Admonitions Predefinidos}

\begin{sphinxadmonition}{note}{Nota:}
\sphinxAtStartPar
Este es un admonition de tipo «note».
\end{sphinxadmonition}

\begin{sphinxadmonition}{warning}{Advertencia:}
\sphinxAtStartPar
Este es un admonition de tipo «warning».
\end{sphinxadmonition}

\begin{sphinxadmonition}{tip}{Truco:}
\sphinxAtStartPar
Este es un admonition de tipo «tip».
\end{sphinxadmonition}
\end{sphinxadmonition}

\sphinxAtStartPar
Esto se logra con:

\begin{sphinxadmonition}{note}{Markdown}

\begin{sphinxVerbatim}[commandchars=\\\{\}]
```\PYGZob{}note\PYGZcb{}
Este es un admonition de tipo \PYGZdq{}note\PYGZdq{}.
```

```\PYGZob{}warning\PYGZcb{}
Este es un admonition de tipo \PYGZdq{}warning\PYGZdq{}.
```

```\PYGZob{}tip\PYGZcb{}
Este es un admonition de tipo \PYGZdq{}tip\PYGZdq{}.
```
\end{sphinxVerbatim}
\end{sphinxadmonition}


\bigskip\hrule\bigskip



\subsection{3.2.3 Admonitions con Títulos Personalizados}
\label{\detokenize{3_guia_myst/advertencias:admonitions-con-titulos-personalizados}}
\sphinxAtStartPar
Puedes agregar un título personalizado a un admonition para especificar su propósito.

\begin{sphinxadmonition}{note}{Ejemplo con Título Personalizado}

\begin{sphinxadmonition}{note}{Recuerda Esto}

\sphinxAtStartPar
Este es un admonition con un título personalizado.
\end{sphinxadmonition}
\end{sphinxadmonition}

\sphinxAtStartPar
Esto se logra con:

\begin{sphinxadmonition}{note}{Markdown}

\begin{sphinxVerbatim}[commandchars=\\\{\}]
```\PYGZob{}admonition\PYGZcb{} Recuerda Esto
Este es un admonition con un título personalizado.
\end{sphinxVerbatim}
\end{sphinxadmonition}


\bigskip\hrule\bigskip



\subsection{3.2.4 Admonitions con Código Formateado}
\label{\detokenize{3_guia_myst/advertencias:admonitions-con-codigo-formateado}}
\sphinxAtStartPar
Puedes incluir bloques de código dentro de un admonition para combinar explicaciones con ejemplos.

\begin{sphinxadmonition}{note}{Ejemplo Admonition con Código}

\begin{sphinxadmonition}{warning}{Advertencia:}\subsubsection*{Importante}

\sphinxAtStartPar
Cuidado con los índices fuera de rango en Python:

\begin{sphinxVerbatim}[commandchars=\\\{\}]
\PYG{n}{lista} \PYG{o}{=} \PYG{p}{[}\PYG{l+m+mi}{1}\PYG{p}{,} \PYG{l+m+mi}{2}\PYG{p}{,} \PYG{l+m+mi}{3}\PYG{p}{]}
\PYG{n+nb}{print}\PYG{p}{(}\PYG{n}{lista}\PYG{p}{[}\PYG{l+m+mi}{3}\PYG{p}{]}\PYG{p}{)}  \PYG{c+c1}{\PYGZsh{} Esto genera un IndexError}
\end{sphinxVerbatim}
\end{sphinxadmonition}
\end{sphinxadmonition}

\sphinxAtStartPar
Esto se logra con:

\begin{sphinxadmonition}{note}{Markdown}

\begin{sphinxVerbatim}[commandchars=\\\{\}]
````\PYGZob{}warning\PYGZcb{}
\PYGZsh{}\PYGZsh{}\PYGZsh{} Importante
Cuidado con los índices fuera de rango en Python:

```python
lista = [1, 2, 3]
print(lista[3])  \PYGZsh{} Esto genera un IndexError
\end{sphinxVerbatim}
\end{sphinxadmonition}


\bigskip\hrule\bigskip



\subsection{3.2.5 Personalización Avanzada}
\label{\detokenize{3_guia_myst/advertencias:personalizacion-avanzada}}
\sphinxAtStartPar
Puedes personalizar la apariencia de los admonitions con estilos adicionales usando etiquetas HTML o CSS dentro de ellos.

\begin{sphinxadmonition}{note}{Ejemplo de Personalización}

\begin{sphinxadmonition}{tip}{Truco:}
\sphinxAtStartPar
Este es un admonition personalizado con texto verde y negrita.
\end{sphinxadmonition}
\end{sphinxadmonition}

\sphinxAtStartPar
Esto se logra con:

\begin{sphinxadmonition}{note}{Markdown}

\begin{sphinxVerbatim}[commandchars=\\\{\}]
```\PYGZob{}tip\PYGZcb{}
\PYGZlt{}span style=\PYGZdq{}color: green; font\PYGZhy{}weight: bold;\PYGZdq{}\PYGZgt{}Este es un admonition personalizado con texto verde y negrita.\PYGZlt{}/span\PYGZgt{}
\end{sphinxVerbatim}
\end{sphinxadmonition}

\sphinxstepscope


\section{3.3 Imágenes y Figuras en MyST}
\label{\detokenize{3_guia_myst/imagenes_figuras:imagenes-y-figuras-en-myst}}\label{\detokenize{3_guia_myst/imagenes_figuras::doc}}

\subsection{3.3.1 Imágenes Básicas}
\label{\detokenize{3_guia_myst/imagenes_figuras:imagenes-basicas}}
\sphinxAtStartPar
Para insertar imágenes básicas en MyST, utiliza la sintaxis de Markdown estándar con \sphinxcode{\sphinxupquote{!{[}{]}()}}.

\begin{sphinxadmonition}{note}{Ejemplo de Imagen Básica}

\sphinxAtStartPar
\sphinxincludegraphics{{fun-fish}.png}
\end{sphinxadmonition}

\sphinxAtStartPar
Esto se logra con:

\begin{sphinxadmonition}{note}{Markdown}

\begin{sphinxVerbatim}[commandchars=\\\{\}]
![Descripción de la imagen](../\PYGZus{}static/fun\PYGZhy{}fish.png \PYGZdq{}Fun Fish is Fun\PYGZdq{})
\end{sphinxVerbatim}
\end{sphinxadmonition}


\bigskip\hrule\bigskip



\subsection{3.3.2 Figuras con MyST}
\label{\detokenize{3_guia_myst/imagenes_figuras:figuras-con-myst}}
\sphinxAtStartPar
Las figuras en MyST son bloques enriquecidos que permiten agregar leyendas y configuraciones adicionales a las imágenes.

\begin{sphinxadmonition}{note}{Ejemplo de Figura con Leyenda}

\begin{figure}[H]
\centering
\capstart

\noindent\sphinxincludegraphics{{fun-fish}.png}
\caption{Este es un ejemplo de figura con una leyenda centrada.}\label{\detokenize{3_guia_myst/imagenes_figuras:pescado-divertido}}\end{figure}
\end{sphinxadmonition}

\sphinxAtStartPar
Esto se logra con:

\begin{sphinxadmonition}{note}{Markdown}

\begin{sphinxVerbatim}[commandchars=\\\{\}]
```\PYGZob{}figure\PYGZcb{} ../\PYGZus{}static/fun\PYGZhy{}fish.png
\PYGZhy{}\PYGZhy{}\PYGZhy{}
align: center
name: pescado\PYGZhy{}divertido
\PYGZhy{}\PYGZhy{}\PYGZhy{}
Este es un ejemplo de figura con una leyenda centrada.
\end{sphinxVerbatim}
\end{sphinxadmonition}


\bigskip\hrule\bigskip



\subsection{3.3.3 Ajustes de Alineación}
\label{\detokenize{3_guia_myst/imagenes_figuras:ajustes-de-alineacion}}
\sphinxAtStartPar
Puedes controlar la alineación de las imágenes o figuras usando el parámetro \sphinxcode{\sphinxupquote{align}}.

\begin{sphinxadmonition}{note}{Ejemplo de Alineación}


\begin{wrapfigure}{l}{0pt}
\centering
\noindent\sphinxincludegraphics{{fun-fish}.png}
\caption{Figura alineada a la izquierda.}\label{\detokenize{3_guia_myst/imagenes_figuras:id1}}\end{wrapfigure}

\mbox{}\par\vskip-\dimexpr\baselineskip+\parskip\relax


\begin{wrapfigure}{r}{0pt}
\centering
\noindent\sphinxincludegraphics{{fun-fish}.png}
\caption{Figura alineada a la derecha.}\label{\detokenize{3_guia_myst/imagenes_figuras:id2}}\end{wrapfigure}

\mbox{}\par\vskip-\dimexpr\baselineskip+\parskip\relax

\begin{figure}[H]
\centering
\capstart

\noindent\sphinxincludegraphics{{fun-fish}.png}
\caption{Figura centrada.}\label{\detokenize{3_guia_myst/imagenes_figuras:id3}}\end{figure}
\end{sphinxadmonition}

\sphinxAtStartPar
Esto se logra con:

\begin{sphinxadmonition}{note}{Markdown}

\begin{sphinxVerbatim}[commandchars=\\\{\}]
```\PYGZob{}figure\PYGZcb{} ../\PYGZus{}static/fun\PYGZhy{}fish.png
\PYGZhy{}\PYGZhy{}\PYGZhy{}
align: left
\PYGZhy{}\PYGZhy{}\PYGZhy{}
Figura alineada a la izquierda.
```
```\PYGZob{}figure\PYGZcb{} ../\PYGZus{}static/fun\PYGZhy{}fish.png
\PYGZhy{}\PYGZhy{}\PYGZhy{}
align: right
\PYGZhy{}\PYGZhy{}\PYGZhy{}
Figura alineada a la derecha.
```
```\PYGZob{}figure\PYGZcb{} ../\PYGZus{}static/fun\PYGZhy{}fish.png
\PYGZhy{}\PYGZhy{}\PYGZhy{}
align: center
\PYGZhy{}\PYGZhy{}\PYGZhy{}
Figura centrada.
```


\end{sphinxVerbatim}
\end{sphinxadmonition}


\bigskip\hrule\bigskip



\subsection{3.3.4 Escalado de Imágenes}
\label{\detokenize{3_guia_myst/imagenes_figuras:escalado-de-imagenes}}
\sphinxAtStartPar
Puedes ajustar el tamaño de las imágenes con el parámetro \sphinxcode{\sphinxupquote{width}}.

\begin{sphinxadmonition}{note}{Ejemplo de Escalado}

\begin{figure}[H]
\centering
\capstart

\noindent\sphinxincludegraphics[width=0.500\linewidth]{{fun-fish}.png}
\caption{Figura con un ancho del 50\%.}\label{\detokenize{3_guia_myst/imagenes_figuras:id4}}\end{figure}

\begin{figure}[H]
\centering
\capstart

\noindent\sphinxincludegraphics[width=200\sphinxpxdimen]{{fun-fish}.png}
\caption{Figura con un ancho de 200px.}\label{\detokenize{3_guia_myst/imagenes_figuras:id5}}\end{figure}
\end{sphinxadmonition}

\sphinxAtStartPar
Esto se logra con:

\begin{sphinxadmonition}{note}{Markdown}

\begin{sphinxVerbatim}[commandchars=\\\{\}]
```\PYGZob{}figure\PYGZcb{} ../\PYGZus{}static/fun\PYGZhy{}fish.png
\PYGZhy{}\PYGZhy{}\PYGZhy{}
width: 50\PYGZpc{}
\PYGZhy{}\PYGZhy{}\PYGZhy{}
Figura con un ancho del 50\PYGZpc{}.
```

```\PYGZob{}figure\PYGZcb{} ../\PYGZus{}static/fun\PYGZhy{}fish.png
\PYGZhy{}\PYGZhy{}\PYGZhy{}
width: 200px
\PYGZhy{}\PYGZhy{}\PYGZhy{}
Figura con un ancho de 200px.
\end{sphinxVerbatim}
\end{sphinxadmonition}


\bigskip\hrule\bigskip



\subsection{3.3.5 Referencias a Figuras}
\label{\detokenize{3_guia_myst/imagenes_figuras:referencias-a-figuras}}
\sphinxAtStartPar
Puedes asignar un nombre a las figuras con el parámetro \sphinxcode{\sphinxupquote{name}} y luego referenciarlas en el texto.

\begin{sphinxadmonition}{note}{Ejemplo de Referencias a Figuras}

\begin{figure}[H]
\centering
\capstart

\noindent\sphinxincludegraphics{{fun-fish}.png}
\caption{Figura con nombre para referencia.}\label{\detokenize{3_guia_myst/imagenes_figuras:fun-fish}}\end{figure}
\end{sphinxadmonition}

\sphinxAtStartPar
Como se muestra en {\hyperref[\detokenize{3_guia_myst/imagenes_figuras:fun-fish}]{\sphinxcrossref{\DUrole{std}{\DUrole{std-ref}{Figura con nombre para referencia.}}}}}, puedes referenciar figuras en tu documento.

\sphinxAtStartPar
Esto se logra con:

\begin{sphinxadmonition}{note}{Markdown}

\begin{sphinxVerbatim}[commandchars=\\\{\}]
```\PYGZob{}figure\PYGZcb{} ../\PYGZus{}static/fun\PYGZhy{}fish.png
\PYGZhy{}\PYGZhy{}\PYGZhy{}
name: fun\PYGZhy{}fish
\PYGZhy{}\PYGZhy{}\PYGZhy{}
Figura con nombre para referencia.
```

Como se muestra en \PYGZob{}ref\PYGZcb{}`fun\PYGZhy{}fish`, puedes referenciar figuras en tu documento.
\end{sphinxVerbatim}
\end{sphinxadmonition}

\sphinxstepscope


\section{3.4 Tablas en MyST}
\label{\detokenize{3_guia_myst/tablas:tablas-en-myst}}\label{\detokenize{3_guia_myst/tablas::doc}}

\subsection{3.4.1 Tablas Básicas}
\label{\detokenize{3_guia_myst/tablas:tablas-basicas}}
\sphinxAtStartPar
Las tablas en MyST se pueden crear usando la sintaxis básica de Markdown.

\begin{sphinxadmonition}{note}{Ejemplo de Tabla Básica}


\begin{savenotes}\sphinxattablestart
\sphinxthistablewithglobalstyle
\centering
\begin{tabulary}{\linewidth}[t]{TTT}
\sphinxtoprule
\sphinxstyletheadfamily 
\sphinxAtStartPar
Columna 1
&\sphinxstyletheadfamily 
\sphinxAtStartPar
Columna 2
&\sphinxstyletheadfamily 
\sphinxAtStartPar
Columna 3
\\
\sphinxmidrule
\sphinxtableatstartofbodyhook
\sphinxAtStartPar
Fila 1
&
\sphinxAtStartPar
Valor 1
&
\sphinxAtStartPar
Valor 2
\\
\sphinxhline
\sphinxAtStartPar
Fila 2
&
\sphinxAtStartPar
Valor 3
&
\sphinxAtStartPar
Valor 4
\\
\sphinxbottomrule
\end{tabulary}
\sphinxtableafterendhook\par
\sphinxattableend\end{savenotes}
\end{sphinxadmonition}

\sphinxAtStartPar
Esto se logra con:

\begin{sphinxadmonition}{note}{Markdown}

\begin{sphinxVerbatim}[commandchars=\\\{\}]
\PYG{o}{|} \PYG{n}{Columna} \PYG{l+m+mi}{1} \PYG{o}{|} \PYG{n}{Columna} \PYG{l+m+mi}{2} \PYG{o}{|} \PYG{n}{Columna} \PYG{l+m+mi}{3} \PYG{o}{|}
\PYG{o}{|}\PYG{o}{\PYGZhy{}}\PYG{o}{\PYGZhy{}}\PYG{o}{\PYGZhy{}}\PYG{o}{\PYGZhy{}}\PYG{o}{\PYGZhy{}}\PYG{o}{\PYGZhy{}}\PYG{o}{\PYGZhy{}}\PYG{o}{\PYGZhy{}}\PYG{o}{\PYGZhy{}}\PYG{o}{\PYGZhy{}}\PYG{o}{\PYGZhy{}}\PYG{o}{|}\PYG{o}{\PYGZhy{}}\PYG{o}{\PYGZhy{}}\PYG{o}{\PYGZhy{}}\PYG{o}{\PYGZhy{}}\PYG{o}{\PYGZhy{}}\PYG{o}{\PYGZhy{}}\PYG{o}{\PYGZhy{}}\PYG{o}{\PYGZhy{}}\PYG{o}{\PYGZhy{}}\PYG{o}{\PYGZhy{}}\PYG{o}{\PYGZhy{}}\PYG{o}{|}\PYG{o}{\PYGZhy{}}\PYG{o}{\PYGZhy{}}\PYG{o}{\PYGZhy{}}\PYG{o}{\PYGZhy{}}\PYG{o}{\PYGZhy{}}\PYG{o}{\PYGZhy{}}\PYG{o}{\PYGZhy{}}\PYG{o}{\PYGZhy{}}\PYG{o}{\PYGZhy{}}\PYG{o}{\PYGZhy{}}\PYG{o}{\PYGZhy{}}\PYG{o}{|}
\PYG{o}{|} \PYG{n}{Fila} \PYG{l+m+mi}{1}    \PYG{o}{|} \PYG{n}{Valor} \PYG{l+m+mi}{1}   \PYG{o}{|} \PYG{n}{Valor} \PYG{l+m+mi}{2}   \PYG{o}{|}
\PYG{o}{|} \PYG{n}{Fila} \PYG{l+m+mi}{2}    \PYG{o}{|} \PYG{n}{Valor} \PYG{l+m+mi}{3}   \PYG{o}{|} \PYG{n}{Valor} \PYG{l+m+mi}{4}   \PYG{o}{|}
\end{sphinxVerbatim}
\end{sphinxadmonition}


\bigskip\hrule\bigskip



\subsection{3.4.2 Tablas Mejoradas con MyST}
\label{\detokenize{3_guia_myst/tablas:tablas-mejoradas-con-myst}}
\sphinxAtStartPar
Puedes mejorar tus tablas utilizando directivas de MyST para agregar alineaciones, anchos y estilos avanzados.

\begin{sphinxadmonition}{note}{Ejemplo de Tabla Mejorada}


\begin{savenotes}\sphinxattablestart
\sphinxthistablewithglobalstyle
\centering
\begin{tabulary}{\linewidth}[t]{TTT}
\sphinxtoprule
\sphinxstyletheadfamily 
\sphinxAtStartPar
Cabecera 1
&\sphinxstyletheadfamily 
\sphinxAtStartPar
Cabecera 2
&\sphinxstyletheadfamily 
\sphinxAtStartPar
Cabecera 3
\\
\sphinxmidrule
\sphinxtableatstartofbodyhook
\sphinxAtStartPar
Fila 1, Col 1
&
\sphinxAtStartPar
Fila 1, Col 2
&
\sphinxAtStartPar
Fila 1, Col 3
\\
\sphinxhline
\sphinxAtStartPar
Fila 2, Col 1
&
\sphinxAtStartPar
Fila 2, Col 2
&
\sphinxAtStartPar
Fila 2, Col 3
\\
\sphinxbottomrule
\end{tabulary}
\sphinxtableafterendhook\par
\sphinxattableend\end{savenotes}
\end{sphinxadmonition}

\sphinxAtStartPar
Esto se logra con:

\begin{sphinxadmonition}{note}{Markdown}

\begin{sphinxVerbatim}[commandchars=\\\{\}]
```\PYGZob{}list\PYGZhy{}table\PYGZcb{}
\PYGZhy{}\PYGZhy{}\PYGZhy{}
header\PYGZhy{}rows: 1
\PYGZhy{}\PYGZhy{}\PYGZhy{}
* \PYGZhy{} Cabecera 1
  \PYGZhy{} Cabecera 2
  \PYGZhy{} Cabecera 3
* \PYGZhy{} Fila 1, Col 1
  \PYGZhy{} Fila 1, Col 2
  \PYGZhy{} Fila 1, Col 3
* \PYGZhy{} Fila 2, Col 1
  \PYGZhy{} Fila 2, Col 2
  \PYGZhy{} Fila 2, Col 3
\end{sphinxVerbatim}
\end{sphinxadmonition}


\bigskip\hrule\bigskip



\subsection{3.4.3 Tablas con Ancho Personalizado}
\label{\detokenize{3_guia_myst/tablas:tablas-con-ancho-personalizado}}
\sphinxAtStartPar
Puedes ajustar el ancho de las columnas para que se adapten mejor al contenido.

\begin{sphinxadmonition}{note}{Ejemplo de Ancho Personalizado}


\begin{savenotes}\sphinxattablestart
\sphinxthistablewithglobalstyle
\centering
\begin{tabular}[t]{\X{20}{100}\X{30}{100}\X{50}{100}}
\sphinxtoprule
\sphinxstyletheadfamily 
\sphinxAtStartPar
Columna Estrecha
&\sphinxstyletheadfamily 
\sphinxAtStartPar
Columna Mediana
&\sphinxstyletheadfamily 
\sphinxAtStartPar
Columna Ancha
\\
\sphinxmidrule
\sphinxtableatstartofbodyhook
\sphinxAtStartPar
Corto
&
\sphinxAtStartPar
Texto de longitud media
&
\sphinxAtStartPar
Este texto es más largo para demostrar el ancho personalizado
\\
\sphinxbottomrule
\end{tabular}
\sphinxtableafterendhook\par
\sphinxattableend\end{savenotes}
\end{sphinxadmonition}

\sphinxAtStartPar
Esto se logra con:

\begin{sphinxadmonition}{note}{Markdown}

\begin{sphinxVerbatim}[commandchars=\\\{\}]
```\PYGZob{}list\PYGZhy{}table\PYGZcb{}
\PYGZhy{}\PYGZhy{}\PYGZhy{}
header\PYGZhy{}rows: 1
widths: 20 30 50
\PYGZhy{}\PYGZhy{}\PYGZhy{}
* \PYGZhy{} Columna Estrecha
  \PYGZhy{} Columna Mediana
  \PYGZhy{} Columna Ancha
* \PYGZhy{} Corto
  \PYGZhy{} Texto de longitud media
  \PYGZhy{} Este texto es más largo para demostrar el ancho personalizado
\end{sphinxVerbatim}
\end{sphinxadmonition}


\bigskip\hrule\bigskip



\subsection{3.4.4 Combinar Celdas en Tablas}
\label{\detokenize{3_guia_myst/tablas:combinar-celdas-en-tablas}}
\sphinxAtStartPar
No es posible combinar celdas directamente con la sintaxis de Markdown o MyST, pero puedes usar HTML dentro de MyST si necesitas esta funcionalidad.

\begin{sphinxadmonition}{note}{Ejemplo de Tabla con Celdas Combinadas}


\end{sphinxadmonition}

\sphinxAtStartPar
Esto se logra con:

\begin{sphinxadmonition}{note}{Markdown}

\begin{sphinxVerbatim}[commandchars=\\\{\}]
```\PYGZob{}raw\PYGZcb{} html
\PYGZlt{}table\PYGZgt{}
  \PYGZlt{}tr\PYGZgt{}
    \PYGZlt{}th\PYGZgt{}Cabecera 1\PYGZlt{}/th\PYGZgt{}
    \PYGZlt{}th\PYGZgt{}Cabecera 2\PYGZlt{}/th\PYGZgt{}
    \PYGZlt{}th\PYGZgt{}Cabecera 3\PYGZlt{}/th\PYGZgt{}
  \PYGZlt{}/tr\PYGZgt{}
  \PYGZlt{}tr\PYGZgt{}
    \PYGZlt{}td rowspan=\PYGZdq{}2\PYGZdq{}\PYGZgt{}Celda combinada\PYGZlt{}/td\PYGZgt{}
    \PYGZlt{}td\PYGZgt{}Fila 1, Col 2\PYGZlt{}/td\PYGZgt{}
    \PYGZlt{}td\PYGZgt{}Fila 1, Col 3\PYGZlt{}/td\PYGZgt{}
  \PYGZlt{}/tr\PYGZgt{}
  \PYGZlt{}tr\PYGZgt{}
    \PYGZlt{}td\PYGZgt{}Fila 2, Col 2\PYGZlt{}/td\PYGZgt{}
    \PYGZlt{}td\PYGZgt{}Fila 2, Col 3\PYGZlt{}/td\PYGZgt{}
  \PYGZlt{}/tr\PYGZgt{}
\PYGZlt{}/table\PYGZgt{}
\end{sphinxVerbatim}
\end{sphinxadmonition}

\sphinxstepscope


\section{3.5 Códigos Fuente y APIs en MyST}
\label{\detokenize{3_guia_myst/code_api:codigos-fuente-y-apis-en-myst}}\label{\detokenize{3_guia_myst/code_api::doc}}

\subsection{3.5.1 Bloques de Código}
\label{\detokenize{3_guia_myst/code_api:bloques-de-codigo}}
\sphinxAtStartPar
Para mostrar bloques de código en MyST, puedes usar la sintaxis estándar de Markdown con tres acentos graves \textasciigrave{}\textasciigrave{}\textasciigrave{}\textasciigrave{}\textasciigrave{} o usar directivas enriquecidas para mayor control.

\begin{sphinxadmonition}{note}{Ejemplo de Bloque de Código Básico}

\begin{sphinxVerbatim}[commandchars=\\\{\}]
\PYG{c+c1}{\PYGZsh{} Este es un bloque de código básico en Python}
\PYG{n+nb}{print}\PYG{p}{(}\PYG{l+s+s2}{\PYGZdq{}}\PYG{l+s+s2}{Hola, mundo!}\PYG{l+s+s2}{\PYGZdq{}}\PYG{p}{)}
\end{sphinxVerbatim}
\end{sphinxadmonition}

\sphinxAtStartPar
Esto se logra con:

\begin{sphinxadmonition}{note}{Markdown}

\begin{sphinxVerbatim}[commandchars=\\\{\}]
```python
\PYGZsh{} Este es un bloque de código básico en Python
print(\PYGZdq{}Hola, mundo!\PYGZdq{})
\end{sphinxVerbatim}
\end{sphinxadmonition}


\bigskip\hrule\bigskip



\subsection{3.5.2 Bloques de Código con Directivas}
\label{\detokenize{3_guia_myst/code_api:bloques-de-codigo-con-directivas}}
\sphinxAtStartPar
Las directivas de MyST permiten personalizar aún más los bloques de código, añadiendo títulos, líneas resaltadas, etc.

\begin{sphinxadmonition}{note}{Ejemplo de Bloque de Código con Directivas}

\def\sphinxLiteralBlockLabel{\label{\detokenize{3_guia_myst/code_api:ejemplo-de-codigo-python}}}
\fvset{hllines={, 2,}}%
\begin{sphinxVerbatim}[commandchars=\\\{\},numbers=left,firstnumber=1,stepnumber=1]
\PYG{c+c1}{\PYGZsh{} Código Python con configuración personalizada}
\PYG{n}{x} \PYG{o}{=} \PYG{l+m+mi}{10}
\PYG{n+nb}{print}\PYG{p}{(}\PYG{l+s+sa}{f}\PYG{l+s+s2}{\PYGZdq{}}\PYG{l+s+s2}{El valor de x es }\PYG{l+s+si}{\PYGZob{}}\PYG{n}{x}\PYG{l+s+si}{\PYGZcb{}}\PYG{l+s+s2}{\PYGZdq{}}\PYG{p}{)}
\end{sphinxVerbatim}
\sphinxresetverbatimhllines
\end{sphinxadmonition}

\sphinxAtStartPar
Esto se logra con:

\begin{sphinxadmonition}{note}{Markdown}

\begin{sphinxVerbatim}[commandchars=\\\{\}]
```\PYGZob{}code\PYGZhy{}block\PYGZcb{} python
\PYGZhy{}\PYGZhy{}\PYGZhy{}
name: Ejemplo de Código Python
linenos: true
emphasize\PYGZhy{}lines: 2
\PYGZhy{}\PYGZhy{}\PYGZhy{}
\PYGZsh{} Código Python con configuración personalizada
x = 10
print(f\PYGZdq{}El valor de x es \PYGZob{}x\PYGZcb{}\PYGZdq{})
\end{sphinxVerbatim}
\end{sphinxadmonition}


\bigskip\hrule\bigskip



\subsection{3.5.3 Incrustar APIs en Documentos}
\label{\detokenize{3_guia_myst/code_api:incrustar-apis-en-documentos}}
\sphinxAtStartPar
Puedes incluir documentaciones de APIs directamente en tus documentos usando la directiva \sphinxcode{\sphinxupquote{automodule}} o \sphinxcode{\sphinxupquote{autoclass}} de Sphinx.

\begin{sphinxadmonition}{note}{Markdown}

\begin{sphinxVerbatim}[commandchars=\\\{\}]
```\PYGZob{}automodule\PYGZcb{} mi\PYGZus{}modulo.ejemplo
\PYGZhy{}\PYGZhy{}\PYGZhy{}
members: true
undoc\PYGZhy{}members: true
show\PYGZhy{}inheritance: true
\PYGZhy{}\PYGZhy{}\PYGZhy{}
\end{sphinxVerbatim}
\end{sphinxadmonition}


\bigskip\hrule\bigskip



\subsection{3.5.4 Resaltar Fragmentos de Código}
\label{\detokenize{3_guia_myst/code_api:resaltar-fragmentos-de-codigo}}
\sphinxAtStartPar
Puedes resaltar fragmentos específicos dentro del texto utilizando bloques en línea.

\begin{sphinxadmonition}{note}{Ejemplo de Código en Línea}

\sphinxAtStartPar
Este es un ejemplo de \sphinxcode{\sphinxupquote{código en línea}} dentro de un párrafo.
\end{sphinxadmonition}

\sphinxAtStartPar
Esto se logra con:

\begin{sphinxadmonition}{note}{Markdown}

\begin{sphinxVerbatim}[commandchars=\\\{\}]
Este es un ejemplo de `código en línea` dentro de un párrafo.
\end{sphinxVerbatim}
\end{sphinxadmonition}


\bigskip\hrule\bigskip



\subsection{3.5.5 Adjuntar Archivos de Código}
\label{\detokenize{3_guia_myst/code_api:adjuntar-archivos-de-codigo}}
\sphinxAtStartPar
Para incluir archivos de código externos en tu documentación, utiliza la directiva \sphinxcode{\sphinxupquote{literalinclude}}.

\begin{sphinxadmonition}{note}{Ejemplo de Inclusión de Archivo}

\fvset{hllines={, 3,}}%
\begin{sphinxVerbatim}[commandchars=\\\{\},numbers=left,firstnumber=1,stepnumber=1]
\PYG{k}{def}\PYG{+w}{ }\PYG{n+nf}{suma}\PYG{p}{(}\PYG{n}{a}\PYG{p}{,} \PYG{n}{b}\PYG{p}{)}\PYG{p}{:}
\PYG{+w}{    }\PYG{l+s+sd}{\PYGZdq{}\PYGZdq{}\PYGZdq{}Suma dos números.}

\PYG{l+s+sd}{    Args:}
\PYG{l+s+sd}{        a (float): Primer número.}
\PYG{l+s+sd}{        b (float): Segundo número.}

\PYG{l+s+sd}{    Returns:}
\PYG{l+s+sd}{        float: Resultado de la suma.}
\PYG{l+s+sd}{    \PYGZdq{}\PYGZdq{}\PYGZdq{}}
    \PYG{k}{return} \PYG{n}{a} \PYG{o}{+} \PYG{n}{b}


\PYG{k}{def}\PYG{+w}{ }\PYG{n+nf}{resta}\PYG{p}{(}\PYG{n}{a}\PYG{p}{,} \PYG{n}{b}\PYG{p}{)}\PYG{p}{:}
\PYG{+w}{    }\PYG{l+s+sd}{\PYGZdq{}\PYGZdq{}\PYGZdq{}Resta dos números.}

\PYG{l+s+sd}{    Args:}
\PYG{l+s+sd}{        a (float): Primer número.}
\PYG{l+s+sd}{        b (float): Segundo número.}

\PYG{l+s+sd}{    Returns:}
\PYG{l+s+sd}{        float: Resultado de la resta.}
\PYG{l+s+sd}{    \PYGZdq{}\PYGZdq{}\PYGZdq{}}
    \PYG{k}{return} \PYG{n}{a} \PYG{o}{\PYGZhy{}} \PYG{n}{b}


\PYG{k}{def}\PYG{+w}{ }\PYG{n+nf}{multiplicacion}\PYG{p}{(}\PYG{n}{a}\PYG{p}{,} \PYG{n}{b}\PYG{p}{)}\PYG{p}{:}
\PYG{+w}{    }\PYG{l+s+sd}{\PYGZdq{}\PYGZdq{}\PYGZdq{}Multiplica dos números.}

\PYG{l+s+sd}{    Args:}
\PYG{l+s+sd}{        a (float): Primer número.}
\PYG{l+s+sd}{        b (float): Segundo número.}

\PYG{l+s+sd}{    Returns:}
\PYG{l+s+sd}{        float: Resultado de la multiplicación.}
\PYG{l+s+sd}{    \PYGZdq{}\PYGZdq{}\PYGZdq{}}
    \PYG{k}{return} \PYG{n}{a} \PYG{o}{*} \PYG{n}{b}


\PYG{k}{def}\PYG{+w}{ }\PYG{n+nf}{division}\PYG{p}{(}\PYG{n}{a}\PYG{p}{,} \PYG{n}{b}\PYG{p}{)}\PYG{p}{:}
\PYG{+w}{    }\PYG{l+s+sd}{\PYGZdq{}\PYGZdq{}\PYGZdq{}Divide dos números.}

\PYG{l+s+sd}{    Args:}
\PYG{l+s+sd}{        a (float): Dividendo.}
\PYG{l+s+sd}{        b (float): Divisor. No puede ser cero.}

\PYG{l+s+sd}{    Returns:}
\PYG{l+s+sd}{        float: Resultado de la división.}

\PYG{l+s+sd}{    Raises:}
\PYG{l+s+sd}{        ValueError: Si el divisor es cero.}
\PYG{l+s+sd}{    \PYGZdq{}\PYGZdq{}\PYGZdq{}}
    \PYG{k}{if} \PYG{n}{b} \PYG{o}{==} \PYG{l+m+mi}{0}\PYG{p}{:}
        \PYG{k}{raise} \PYG{n+ne}{ValueError}\PYG{p}{(}\PYG{l+s+s2}{\PYGZdq{}}\PYG{l+s+s2}{El divisor no puede ser cero.}\PYG{l+s+s2}{\PYGZdq{}}\PYG{p}{)}
    \PYG{k}{return} \PYG{n}{a} \PYG{o}{/} \PYG{n}{b}


\PYG{k}{def}\PYG{+w}{ }\PYG{n+nf}{operaciones\PYGZus{}basicas}\PYG{p}{(}\PYG{n}{a}\PYG{p}{,} \PYG{n}{b}\PYG{p}{)}\PYG{p}{:}
\PYG{+w}{    }\PYG{l+s+sd}{\PYGZdq{}\PYGZdq{}\PYGZdq{}Realiza todas las operaciones básicas entre dos números.}

\PYG{l+s+sd}{    Args:}
\PYG{l+s+sd}{        a (float): Primer número.}
\PYG{l+s+sd}{        b (float): Segundo número.}

\PYG{l+s+sd}{    Returns:}
\PYG{l+s+sd}{        dict: Diccionario con los resultados de suma, resta, multiplicación y división.}

\PYG{l+s+sd}{    Raises:}
\PYG{l+s+sd}{        ValueError: Si el divisor es cero al intentar dividir.}
\PYG{l+s+sd}{    \PYGZdq{}\PYGZdq{}\PYGZdq{}}
    \PYG{k}{try}\PYG{p}{:}
        \PYG{k}{return} \PYG{p}{\PYGZob{}}
            \PYG{l+s+s2}{\PYGZdq{}}\PYG{l+s+s2}{suma}\PYG{l+s+s2}{\PYGZdq{}}\PYG{p}{:} \PYG{n}{suma}\PYG{p}{(}\PYG{n}{a}\PYG{p}{,} \PYG{n}{b}\PYG{p}{)}\PYG{p}{,}
            \PYG{l+s+s2}{\PYGZdq{}}\PYG{l+s+s2}{resta}\PYG{l+s+s2}{\PYGZdq{}}\PYG{p}{:} \PYG{n}{resta}\PYG{p}{(}\PYG{n}{a}\PYG{p}{,} \PYG{n}{b}\PYG{p}{)}\PYG{p}{,}
            \PYG{l+s+s2}{\PYGZdq{}}\PYG{l+s+s2}{multiplicacion}\PYG{l+s+s2}{\PYGZdq{}}\PYG{p}{:} \PYG{n}{multiplicacion}\PYG{p}{(}\PYG{n}{a}\PYG{p}{,} \PYG{n}{b}\PYG{p}{)}\PYG{p}{,}
            \PYG{l+s+s2}{\PYGZdq{}}\PYG{l+s+s2}{division}\PYG{l+s+s2}{\PYGZdq{}}\PYG{p}{:} \PYG{n}{division}\PYG{p}{(}\PYG{n}{a}\PYG{p}{,} \PYG{n}{b}\PYG{p}{)}\PYG{p}{,}
        \PYG{p}{\PYGZcb{}}
    \PYG{k}{except} \PYG{n+ne}{ValueError} \PYG{k}{as} \PYG{n}{e}\PYG{p}{:}
        \PYG{k}{return} \PYG{p}{\PYGZob{}}\PYG{l+s+s2}{\PYGZdq{}}\PYG{l+s+s2}{error}\PYG{l+s+s2}{\PYGZdq{}}\PYG{p}{:} \PYG{n+nb}{str}\PYG{p}{(}\PYG{n}{e}\PYG{p}{)}\PYG{p}{\PYGZcb{}}
\end{sphinxVerbatim}
\sphinxresetverbatimhllines
\end{sphinxadmonition}

\sphinxAtStartPar
Esto se logra con:

\begin{sphinxadmonition}{note}{Markdown}

\begin{sphinxVerbatim}[commandchars=\\\{\}]
```\PYGZob{}literalinclude\PYGZcb{} ../../../src/operaciones.py
\PYGZhy{}\PYGZhy{}\PYGZhy{}
language: python
linenos: true
emphasize\PYGZhy{}lines: 3
\PYGZhy{}\PYGZhy{}\PYGZhy{}
\end{sphinxVerbatim}
\end{sphinxadmonition}


\bigskip\hrule\bigskip



\subsection{3.5.6 Notas sobre la Configuración}
\label{\detokenize{3_guia_myst/code_api:notas-sobre-la-configuracion}}
\sphinxAtStartPar
Al usar directivas como \sphinxcode{\sphinxupquote{automodule}} o \sphinxcode{\sphinxupquote{literalinclude}}, asegúrate de que las rutas de los archivos sean correctas y de que la configuración de Sphinx permita el uso de estas directivas. Esto puede requerir ajustes en \sphinxcode{\sphinxupquote{conf.py}} como:

\begin{sphinxVerbatim}[commandchars=\\\{\}]
\PYG{n}{extensions} \PYG{o}{=} \PYG{p}{[}
    \PYG{l+s+s1}{\PYGZsq{}}\PYG{l+s+s1}{sphinx.ext.autodoc}\PYG{l+s+s1}{\PYGZsq{}}\PYG{p}{,}
    \PYG{l+s+s1}{\PYGZsq{}}\PYG{l+s+s1}{sphinx.ext.napoleon}\PYG{l+s+s1}{\PYGZsq{}}\PYG{p}{,}
    \PYG{l+s+s1}{\PYGZsq{}}\PYG{l+s+s1}{sphinx.ext.viewcode}\PYG{l+s+s1}{\PYGZsq{}}\PYG{p}{,}
\PYG{p}{]}
\end{sphinxVerbatim}

\sphinxstepscope


\section{3.6 Cross\sphinxhyphen{}References en MyST}
\label{\detokenize{3_guia_myst/cross_references:cross-references-en-myst}}\label{\detokenize{3_guia_myst/cross_references::doc}}

\subsection{3.6.1 Enlaces Internos Básicos}
\label{\detokenize{3_guia_myst/cross_references:enlaces-internos-basicos}}
\sphinxAtStartPar
Puedes crear enlaces internos en tu documento MyST utilizando la sintaxis estándar de Markdown o referencias explícitas con etiquetas.

\begin{sphinxadmonition}{note}{Ejemplo de Enlace Interno}

\sphinxAtStartPar
Puedes saltar a la sección {[}3.6.2 Enlaces con Títulos{]}(\#3.6.2 Enlaces\sphinxhyphen{}con\sphinxhyphen{}titulos).
\end{sphinxadmonition}

\sphinxAtStartPar
Esto se logra con:

\begin{sphinxadmonition}{note}{Markdown}

\begin{sphinxVerbatim}[commandchars=\\\{\}]
Puedes saltar a la sección [\PYG{n+nt}{3.6.2 Enlaces con Títulos}](\PYG{n+na}{\PYGZsh{}enlaces\PYGZhy{}con\PYGZhy{}titulos}).
\end{sphinxVerbatim}
\end{sphinxadmonition}


\bigskip\hrule\bigskip



\subsection{3.6.2 Enlaces con Títulos}
\label{\detokenize{3_guia_myst/cross_references:enlaces-con-titulos}}
\sphinxAtStartPar
Los encabezados en MyST automáticamente generan identificadores que pueden ser referenciados mediante \sphinxcode{\sphinxupquote{\#}} seguido del identificador. Si el encabezado tiene caracteres especiales, estos se reemplazan automáticamente (por ejemplo, espacios por guiones).

\begin{sphinxadmonition}{note}{Ejemplo de Referencia a un Encabezado}

\sphinxAtStartPar
Consulta la sección {[}3.6.1 Enlaces Internos Básicos{]}(\#3.6.1 Enlaces\sphinxhyphen{}internos\sphinxhyphen{}basicos) para más detalles.
\end{sphinxadmonition}

\sphinxAtStartPar
Esto se logra con:

\begin{sphinxadmonition}{note}{Markdown}

\begin{sphinxVerbatim}[commandchars=\\\{\}]
Consulta la sección [\PYG{n+nt}{3.6.1 Enlaces Internos Básicos}](\PYG{n+na}{\PYGZsh{}enlaces\PYGZhy{}internos\PYGZhy{}basicos}) para más detalles.
\end{sphinxVerbatim}
\end{sphinxadmonition}


\bigskip\hrule\bigskip



\subsection{3.6.3 Etiquetas Personalizadas}
\label{\detokenize{3_guia_myst/cross_references:etiquetas-personalizadas}}
\sphinxAtStartPar
Puedes agregar etiquetas personalizadas a las secciones o elementos usando la directiva \sphinxcode{\sphinxupquote{\{ref\}}}. Esto permite crear nombres amigables para las referencias.

\begin{sphinxadmonition}{note}{Ejemplo de Etiqueta Personalizada}

\begin{sphinxVerbatim}[commandchars=\\\{\}]
(seccion\PYGZhy{}personalizada)=
\PYG{g+gh}{\PYGZsh{} Título de la Sección}

Contenido de la sección con una etiqueta personalizada.
\end{sphinxVerbatim}

\sphinxAtStartPar
Y puedes referenciarla con:

\begin{sphinxVerbatim}[commandchars=\\\{\}]
:ref:`Consulta esta sección personalizada \PYGZlt{}seccion\PYGZhy{}personalizada\PYGZgt{}` para obtener más información.
\end{sphinxVerbatim}
\end{sphinxadmonition}


\bigskip\hrule\bigskip



\subsection{3.6.4 Referencias Cruzadas a Figuras, Tablas y Código}
\label{\detokenize{3_guia_myst/cross_references:referencias-cruzadas-a-figuras-tablas-y-codigo}}
\sphinxAtStartPar
Puedes referenciar figuras, tablas y bloques de código usando etiquetas y la directiva \sphinxcode{\sphinxupquote{:ref:}}.

\begin{sphinxadmonition}{note}{Ejemplo de Referencia Cruzada}

\begin{figure}[H]
\centering
\capstart

\noindent\sphinxincludegraphics{{fun-fish}.png}
\caption{Este es un pez divertido.}\label{\detokenize{3_guia_myst/cross_references:figura-divertida}}\end{figure}

\sphinxAtStartPar
Luego, puedes referenciar la figura con:

\begin{sphinxVerbatim}[commandchars=\\\{\}]
Mira la figura en \PYGZob{}ref\PYGZcb{}`figura\PYGZhy{}divertida` para más detalles.
\end{sphinxVerbatim}
\end{sphinxadmonition}

\sphinxAtStartPar
Esto se logra asignando una etiqueta en la figura y referenciándola.


\bigskip\hrule\bigskip



\subsection{3.6.5 Referencias Externas}
\label{\detokenize{3_guia_myst/cross_references:referencias-externas}}
\sphinxAtStartPar
Para enlazar a recursos externos, simplemente usa la sintaxis estándar de Markdown.

\begin{sphinxadmonition}{note}{Ejemplo de Enlace Externo}

\sphinxAtStartPar
Visita \sphinxhref{https://myst-parser.readthedocs.io/}{la documentación de MyST} para más información.
\end{sphinxadmonition}

\sphinxAtStartPar
Esto se logra con:

\begin{sphinxadmonition}{note}{Markdown}

\begin{sphinxVerbatim}[commandchars=\\\{\}]
Visita [\PYG{n+nt}{la documentación de MyST}](\PYG{n+na}{https://myst\PYGZhy{}parser.readthedocs.io/}) para más información.
\end{sphinxVerbatim}
\end{sphinxadmonition}


\bigskip\hrule\bigskip


\sphinxAtStartPar
Con estas herramientas, puedes estructurar tus documentos MyST con enlaces internos, referencias cruzadas y etiquetas personalizadas, mejorando la navegación y claridad de tus contenidos.

\sphinxstepscope


\section{3.7 Organización del Contenido en MyST}
\label{\detokenize{3_guia_myst/organizacion_contenido:organizacion-del-contenido-en-myst}}\label{\detokenize{3_guia_myst/organizacion_contenido::doc}}

\subsection{3.7.1 Secciones y Encabezados}
\label{\detokenize{3_guia_myst/organizacion_contenido:secciones-y-encabezados}}
\sphinxAtStartPar
La organización básica del contenido en MyST se realiza mediante encabezados. Estos se generan con el uso de almohadillas \sphinxcode{\sphinxupquote{\#}}, con un número creciente para indicar la jerarquía.

\begin{sphinxadmonition}{note}{Ejemplo de Encabezados Jerárquicos}

\begin{sphinxVerbatim}[commandchars=\\\{\}]
\PYG{g+gh}{\PYGZsh{} Título Principal}
\PYG{g+gu}{\PYGZsh{}\PYGZsh{} Subtítulo 1}
\PYG{g+gu}{\PYGZsh{}\PYGZsh{}\PYGZsh{} Subtítulo 1.1}
\PYG{g+gu}{\PYGZsh{}\PYGZsh{}\PYGZsh{}\PYGZsh{} Subtítulo 1.1.1}
\end{sphinxVerbatim}
\end{sphinxadmonition}

\sphinxAtStartPar
Esto permite estructurar claramente el contenido en niveles.


\bigskip\hrule\bigskip



\subsection{3.7.2 Separadores de Temas}
\label{\detokenize{3_guia_myst/organizacion_contenido:separadores-de-temas}}
\sphinxAtStartPar
Los separadores \sphinxcode{\sphinxupquote{\sphinxhyphen{}\sphinxhyphen{}\sphinxhyphen{}}} ayudan a dividir secciones visualmente, indicando cambios de tema o contenido.

\begin{sphinxadmonition}{note}{Ejemplo de Separador}

\sphinxAtStartPar
Este es el contenido antes del separador.


\bigskip\hrule\bigskip


\sphinxAtStartPar
Este es el contenido después del separador.
\end{sphinxadmonition}

\sphinxAtStartPar
Esto se escribe como:

\begin{sphinxadmonition}{note}{Markdown}

\begin{sphinxVerbatim}[commandchars=\\\{\}]
\PYG{n}{Este} \PYG{n}{es} \PYG{n}{el} \PYG{n}{contenido} \PYG{n}{antes} \PYG{k}{del} \PYG{n}{separador}\PYG{o}{.}

\PYG{o}{\PYGZhy{}}\PYG{o}{\PYGZhy{}}\PYG{o}{\PYGZhy{}}

\PYG{n}{Este} \PYG{n}{es} \PYG{n}{el} \PYG{n}{contenido} \PYG{n}{después} \PYG{k}{del} \PYG{n}{separador}\PYG{o}{.}
\end{sphinxVerbatim}
\end{sphinxadmonition}


\bigskip\hrule\bigskip



\subsection{3.7.3 Archivos Incluidos}
\label{\detokenize{3_guia_myst/organizacion_contenido:archivos-incluidos}}
\sphinxAtStartPar
Para incluir contenido de otros archivos dentro de un documento, utiliza la directiva \sphinxcode{\sphinxupquote{include}}.

\begin{sphinxadmonition}{note}{Ejemplo de Inclusión de Archivo}

\begin{sphinxVerbatim}[commandchars=\\\{\}]
```\PYGZob{}include\PYGZcb{} ../path/to/otro\PYGZus{}documento.md
:relative\PYGZhy{}docs: docs/
:relative\PYGZhy{}images:
```
\end{sphinxVerbatim}
\end{sphinxadmonition}

\sphinxAtStartPar
Esto inserta el contenido del archivo especificado directamente en la posición del documento actual.


\bigskip\hrule\bigskip



\subsection{3.7.4 Índices y Tablas de Contenido}
\label{\detokenize{3_guia_myst/organizacion_contenido:indices-y-tablas-de-contenido}}
\sphinxAtStartPar
Puedes generar automáticamente un índice o tabla de contenido utilizando la directiva \sphinxcode{\sphinxupquote{toctree}}.

\begin{sphinxadmonition}{note}{Ejemplo de Tabla de Contenidos}

\begin{sphinxVerbatim}[commandchars=\\\{\}]
```\PYGZob{}toctree\PYGZcb{}
:maxdepth: 2
:hidden:

seccion1
seccion2
subdirectorio/seccion3
```
\end{sphinxVerbatim}
\end{sphinxadmonition}

\sphinxAtStartPar
Esto genera una tabla de contenido basada en los archivos especificados.


\bigskip\hrule\bigskip



\subsection{3.7.5 Etiquetas y Referencias}
\label{\detokenize{3_guia_myst/organizacion_contenido:etiquetas-y-referencias}}
\sphinxAtStartPar
Organiza contenido utilizando etiquetas personalizadas para referenciar secciones específicas. Usa la directiva \sphinxcode{\sphinxupquote{(identificador)=}} para asignar una etiqueta.

\begin{sphinxadmonition}{note}{Ejemplo de Etiqueta y Referencia}

\begin{sphinxVerbatim}[commandchars=\\\{\}]
(etiqueta\PYGZhy{}seccion)=
\PYG{g+gh}{\PYGZsh{} Título de la Sección}

Contenido de la sección etiquetada.

Consulta la sección \PYGZob{}ref\PYGZcb{}`etiqueta\PYGZhy{}seccion` para más información.
\end{sphinxVerbatim}
\end{sphinxadmonition}

\sphinxstepscope


\section{3.8 Extensiones de Sintaxis en MyST}
\label{\detokenize{3_guia_myst/extensiones:extensiones-de-sintaxis-en-myst}}\label{\detokenize{3_guia_myst/extensiones::doc}}

\subsection{3.8.1 Extensiones Opcionales}
\label{\detokenize{3_guia_myst/extensiones:extensiones-opcionales}}
\sphinxAtStartPar
El parser de MyST incluye varias extensiones opcionales que permiten agregar características avanzadas al contenido. Estas extensiones se pueden habilitar en la configuración de Sphinx mediante la clave \sphinxcode{\sphinxupquote{myst\_enable\_extensions}}.

\sphinxAtStartPar
Ejemplo de configuración en \sphinxcode{\sphinxupquote{conf.py}}:

\begin{sphinxVerbatim}[commandchars=\\\{\}]
\PYG{n}{myst\PYGZus{}enable\PYGZus{}extensions} \PYG{o}{=} \PYG{p}{[}
    \PYG{l+s+s2}{\PYGZdq{}}\PYG{l+s+s2}{dollarmath}\PYG{l+s+s2}{\PYGZdq{}}\PYG{p}{,}
    \PYG{l+s+s2}{\PYGZdq{}}\PYG{l+s+s2}{tasklist}\PYG{l+s+s2}{\PYGZdq{}}\PYG{p}{,}
    \PYG{l+s+s2}{\PYGZdq{}}\PYG{l+s+s2}{fieldlist}\PYG{l+s+s2}{\PYGZdq{}}
\PYG{p}{]}
\end{sphinxVerbatim}

\sphinxAtStartPar
A continuación, se explican tres extensiones con ejemplos:


\bigskip\hrule\bigskip



\subsection{3.8.2 Matemáticas con \sphinxstylestrong{dollarmath}}
\label{\detokenize{3_guia_myst/extensiones:matematicas-con-dollarmath}}
\sphinxAtStartPar
Permite escribir expresiones matemáticas en línea usando \sphinxcode{\sphinxupquote{\$}} para ecuaciones simples o \sphinxcode{\sphinxupquote{\$\$}} para ecuaciones en bloques.

\begin{sphinxadmonition}{note}{Ejemplo de Matemáticas con Dollarmath}

\begin{sphinxVerbatim}[commandchars=\\\{\}]
El área de un círculo es \PYGZdl{}A = \PYGZbs{}pi r\PYGZca{}2\PYGZdl{}.

\PYGZdl{}\PYGZdl{}
E = mc\PYGZca{}2
\PYGZdl{}\PYGZdl{}
\end{sphinxVerbatim}
\end{sphinxadmonition}

\sphinxAtStartPar
Se renderizará como:
\begin{itemize}
\item {} 
\sphinxAtStartPar
En línea: ( A = \textbackslash{}pi r\textasciicircum{}2 )

\item {} 
\sphinxAtStartPar
Bloque:\\
{[}
E = mc\textasciicircum{}2
{]}

\end{itemize}


\bigskip\hrule\bigskip



\subsection{3.8.3 Listas de Tareas con \sphinxstylestrong{tasklist}}
\label{\detokenize{3_guia_myst/extensiones:listas-de-tareas-con-tasklist}}
\sphinxAtStartPar
Esta extensión permite agregar listas de tareas con casillas marcables en Markdown.

\begin{sphinxadmonition}{note}{Ejemplo de Lista de Tareas}

\begin{sphinxVerbatim}[commandchars=\\\{\}]
\PYG{k}{\PYGZhy{} }\PYG{k}{[ ]} Tarea pendiente
\PYG{k}{\PYGZhy{} }\PYG{k}{[x]} Tarea completada
\PYG{k}{\PYGZhy{} }\PYG{k}{[ ]} Otra tarea
\end{sphinxVerbatim}
\end{sphinxadmonition}

\sphinxAtStartPar
Se renderizará como:
\begin{itemize}
\item {} 
\sphinxAtStartPar
{[} {]} Tarea pendiente

\item {} 
\sphinxAtStartPar
{[}x{]} Tarea completada

\item {} 
\sphinxAtStartPar
{[} {]} Otra tarea

\end{itemize}


\bigskip\hrule\bigskip



\subsection{3.8.4 Listas de Campos con \sphinxstylestrong{fieldlist}}
\label{\detokenize{3_guia_myst/extensiones:listas-de-campos-con-fieldlist}}
\sphinxAtStartPar
Esta extensión se utiliza para estructurar definiciones con pares clave\sphinxhyphen{}valor en formato limpio.

\begin{sphinxadmonition}{note}{Ejemplo de Lista de Campos}

\begin{sphinxVerbatim}[commandchars=\\\{\}]
:autor: John Doe
:versión: 1.0
:fecha: 12\PYGZhy{}01\PYGZhy{}2025
\end{sphinxVerbatim}
\end{sphinxadmonition}

\sphinxAtStartPar
Renderiza el contenido como una tabla de definiciones:
\begin{itemize}
\item {} 
\sphinxAtStartPar
\sphinxstylestrong{autor}: John Doe

\item {} 
\sphinxAtStartPar
\sphinxstylestrong{versión}: 1.0

\item {} 
\sphinxAtStartPar
\sphinxstylestrong{fecha}: 12\sphinxhyphen{}01\sphinxhyphen{}2025

\end{itemize}


\bigskip\hrule\bigskip


\sphinxAtStartPar
Habilitar estas extensiones puede enriquecer la presentación y funcionalidad del contenido de tu guía.

\sphinxstepscope


\section{3.9 Roles y Directivas en MyST}
\label{\detokenize{3_guia_myst/roles:roles-y-directivas-en-myst}}\label{\detokenize{3_guia_myst/roles::doc}}

\subsection{3.9.1 ¿Qué son los Roles y las Directivas?}
\label{\detokenize{3_guia_myst/roles:que-son-los-roles-y-las-directivas}}
\sphinxAtStartPar
Los roles son marcadores en línea que agregan formato o funcionalidad, mientras que las directivas son bloques más grandes para incluir contenido dinámico o personalizado.


\bigskip\hrule\bigskip



\subsection{3.9.2 Ejemplo de Roles}
\label{\detokenize{3_guia_myst/roles:ejemplo-de-roles}}

\subsubsection{Rol: \sphinxstylestrong{doc}}
\label{\detokenize{3_guia_myst/roles:rol-doc}}
\sphinxAtStartPar
Permite crear enlaces a otros documentos en tu proyecto.

\begin{sphinxVerbatim}[commandchars=\\\{\}]
Consulta la documentación en \PYGZob{}doc\PYGZcb{}`seccion1`.
\end{sphinxVerbatim}

\sphinxAtStartPar
Renderiza como:
\begin{quote}

\sphinxAtStartPar
Consulta la documentación en \sphinxhref{https://mi-documentacion.com/seccion1}{Sección 1}.
\end{quote}


\bigskip\hrule\bigskip



\subsubsection{Rol: \sphinxstylestrong{ref}}
\label{\detokenize{3_guia_myst/roles:rol-ref}}
\sphinxAtStartPar
Crea referencias a etiquetas definidas previamente.

\begin{sphinxVerbatim}[commandchars=\\\{\}]
Consulta más detalles en \PYGZob{}ref\PYGZcb{}`etiqueta\PYGZhy{}seccion`.
\end{sphinxVerbatim}

\sphinxAtStartPar
Renderiza como:
\begin{quote}

\sphinxAtStartPar
Consulta más detalles en \sphinxhref{https://mi-documentacion.com/seccion\#etiqueta-seccion}{Título de la Sección}.
\end{quote}


\bigskip\hrule\bigskip



\subsubsection{Rol: \sphinxstylestrong{numref}}
\label{\detokenize{3_guia_myst/roles:rol-numref}}
\sphinxAtStartPar
Permite referencias numeradas a figuras, tablas o secciones.

\begin{sphinxVerbatim}[commandchars=\\\{\}]
Consulta la figura \PYGZob{}numref\PYGZcb{}`figura\PYGZhy{}1`.
\end{sphinxVerbatim}


\bigskip\hrule\bigskip



\subsection{3.9.3 Ejemplo de Directivas}
\label{\detokenize{3_guia_myst/roles:ejemplo-de-directivas}}

\subsubsection{Directiva: \sphinxstylestrong{figure}}
\label{\detokenize{3_guia_myst/roles:directiva-figure}}
\sphinxAtStartPar
Incluye una figura con su título y descripción.

\begin{sphinxVerbatim}[commandchars=\\\{\}]
```\PYGZob{}figure\PYGZcb{} ../images/imagen.png
:alt: Una descripción de la imagen
:width: 50\PYGZpc{}

Título de la Figura
\end{sphinxVerbatim}

\begin{sphinxVerbatim}[commandchars=\\\{\}]

\PYG{o}{\PYGZhy{}}\PYG{o}{\PYGZhy{}}\PYG{o}{\PYGZhy{}}

\PYG{c+c1}{\PYGZsh{}\PYGZsh{}\PYGZsh{} Directiva: **`note`**}
\PYG{n}{Destaca} \PYG{n}{contenido} \PYG{n}{importante} \PYG{n}{como} \PYG{n}{una} \PYG{n}{nota}\PYG{o}{.}

\end{sphinxVerbatim}

\begin{sphinxadmonition}{note}{Nota:}
\sphinxAtStartPar
Recuerda actualizar la guía periódicamente.
\end{sphinxadmonition}

\begin{sphinxVerbatim}[commandchars=\\\{\}]

\PYG{n}{Renderiza} \PYG{n}{como}\PYG{p}{:}  
\PYG{o}{\PYGZgt{}} \PYG{o}{*}\PYG{o}{*}\PYG{n}{Nota}\PYG{o}{*}\PYG{o}{*}\PYG{p}{:} \PYG{n}{Recuerda} \PYG{n}{actualizar} \PYG{n}{la} \PYG{n}{guía} \PYG{n}{periódicamente}\PYG{o}{.}

\PYG{o}{\PYGZhy{}}\PYG{o}{\PYGZhy{}}\PYG{o}{\PYGZhy{}}

\PYG{c+c1}{\PYGZsh{}\PYGZsh{}\PYGZsh{} Directiva: **`warning`**}
\PYG{n}{Muestra} \PYG{n}{un} \PYG{n}{mensaje} \PYG{n}{de} \PYG{n}{advertencia}\PYG{o}{.}

\end{sphinxVerbatim}

\begin{sphinxadmonition}{warning}{Advertencia:}
\sphinxAtStartPar
Este proceso puede sobrescribir archivos existentes.
\end{sphinxadmonition}

\begin{sphinxVerbatim}[commandchars=\\\{\}]

\PYG{n}{Renderiza} \PYG{n}{como}\PYG{p}{:}  
\PYG{o}{\PYGZgt{}} \PYGZbs{}\PYG{n}{textwarning} \PYG{o}{*}\PYG{o}{*}\PYG{n}{Advertencia}\PYG{o}{*}\PYG{o}{*}\PYG{p}{:} \PYG{n}{Este} \PYG{n}{proceso} \PYG{n}{puede} \PYG{n}{sobrescribir} \PYG{n}{archivos} \PYG{n}{existentes}\PYG{o}{.}

\PYG{o}{\PYGZhy{}}\PYG{o}{\PYGZhy{}}\PYG{o}{\PYGZhy{}}

\PYG{n}{El} \PYG{n}{uso} \PYG{n}{de} \PYG{n}{roles} \PYG{n}{y} \PYG{n}{directivas} \PYG{n}{en} \PYG{n}{MyST} \PYG{n}{mejora} \PYG{n}{la} \PYG{n}{claridad} \PYG{n}{y} \PYG{n}{navegación} \PYG{k}{del} \PYG{n}{contenido}\PYG{o}{.}
\end{sphinxVerbatim}


\renewcommand{\indexname}{Índice de Módulos Python}
\begin{sphinxtheindex}
\let\bigletter\sphinxstyleindexlettergroup
\bigletter{s}
\item\relax\sphinxstyleindexentry{src.operaciones}\sphinxstyleindexpageref{1_configuracion_inicial/src:\detokenize{module-src.operaciones}}
\end{sphinxtheindex}

\renewcommand{\indexname}{Índice}
\printindex
\end{document}